\chapter{IR and WebAssembly Output of GCD}
\thispagestyle{headings}
\label{chapter:gcd}

%\note{Tobias' proposal that I have an example program and show it at each step of the compilation. Could be a good idea but the code is massive, so this would easily extend to about 10+ pages. Also might require renaming some of the temporaries to make the code easier to read}
%\note{Probably a bit much coming to think of it, so maybe just GCD sample, (untyped-AST?), IR and WebAssembly only to save from having to apply to the chairman of examiners for a longer appendix. I would rename temp variables to make the code easier to understand}

\section{OCaml}
The recursive GCD function as it is defined in the GCD benchmark. The benchmark intentionally uses the subtraction method to compute GCD rather than the modulo method to ensure a larger number of recursive calls.

\begin{figure}[h!]
\begin{minted}[linenos]{OCaml}
let rec gcd a b =
  match (a, b) with
  | (0, y) -> y
  | (x, 0) -> x
  | _ ->
      if a > b then
        gcd (a - b) b
      else
        gcd (b - a) a
\end{minted}
\label{fig:gcd_ocaml}
\end{figure}

\section{Comparison of Optimised IR and WebAssembly}
This section contains extracts from the compiled IR and WebAssembly forms of the above. Note that variable and block names have been modified to improve readability.

\subsection{First Match Case - IR and WebAssembly}
This sample serves to show the differences between the IR and WebAssembly, and the results of optimisations.

\makeatletter
\expandafter\def\csname PYGdefault@tok@err\endcsname{\def\PYGdefault@bc##1{##1}}
\makeatother

\subsubsection*{IR}
\begin{minted}{Perl}
    startblock $match_case_0_y
      setvar int local.$constant_zero_1 0
      binop int ne local.$a_ne_zero local.$a local.$constant_zero_1
      exitblockif $match_case_0_y local.$a_ne_zero
      copyvar int local.$result local.$b
      exitblock $match_block
    endblock $match_case_0_y
\end{minted}

\subsubsection*{WebAssembly}
\begin{minted}[linenos,firstnumber=10]{LISP}
          block $match_case_0_y
            local.get $a
            i32.const 0
            i32.ne
            br_if $match_case_0_y  ; If a is not 0, skip this case
            local.get $b
            local.set $result
            br $match_block  ; Skip over additional cases
          end $match_case_0_y
\end{minted}
Of note here:
\begin{itemize}
\item The stack code generator has eliminated the \textinline{constant_zero_1} and \textinline{a_ne_zero} variables, both of which are computed and then used by the immediately following instruction.
\item The unoptimised code for the match creates a tuple $(a, b)$, and then loads from this tuple. This was eliminated by the tuple-load optimisation and dead-code elimination. The variable $y$ has also been eliminated via copy-propagation, replacing it with $b$.
\end{itemize}

\subsection{Tail-Call Optimisation - WebAssembly Only}
These samples shows how the results of tail-call optimisation show up in compiled WebAssembly.

\begin{minted}[linenos,firstnumber=1]{LISP}
    local.get $a_in
    local.set $a
    local.get $b_in
    local.set $b
    block $break_loop
      loop $continue_loop
        local.get $a
        local.set $a_copy
        block $match_block
\end{minted}
We start by copying the argument variables into `reference variables' a and b, which are references eliminated by ref elimination. Inside the loop, we should copy both $a$ and $b$, however the copy of b was eliminated because we never change b before it is used. In the sample below, we see a `tail-call' being made. $a$ is assigned first, and hence its copy is still needed.
\begin{minted}[linenos,firstnumber=32]{LISP}
            else $if_a_greater_than_b
              local.get $b
              local.get $a
              i32.sub   ; b_minus_a exists only on stack
              local.set $a
              local.get $a_copy ; Note use of a_copy because a was reassigned
              local.set $b
              br $continue_loop
            end $if_a_greater_than_b
\end{minted}