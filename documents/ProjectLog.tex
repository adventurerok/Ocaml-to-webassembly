\documentclass{report}
\usepackage[utf8]{inputenc}

\usepackage[T1]{fontenc}
\usepackage{fancyhdr}
\usepackage[a4paper, total={7in,9in}]{geometry}
\usepackage{amsmath}
\usepackage{listings}
\usepackage{xfrac}

\usepackage{graphicx}
\usepackage{wrapfig}
\graphicspath{ {./graphics/} }

\usepackage{minted}
%\usemintedstyle{colorful}
\setmintedinline{breaklines}

\newcommand{\textinline}{\mintinline{text}}
\newcommand{\cinline}{\mintinline{C}}
\newcommand{\camlinline}{\mintinline{OCaml}}

% Makes all math bold in sections that should be bold
\makeatletter
\g@addto@macro\bfseries{\boldmath}
\makeatother

\newcommand{\thetitle}{Optimising Compiler from OCaml to WebAssembly}

\title{\thetitle}
\author{Paul Durbaba}

\pagestyle{fancy}
\fancyhead{}
\fancyfoot{}

\begin{document}
\maketitle
\tableofcontents

\chapter{PreProposal}

\section{100 words}

I will build a compiler from a subset of OCaml to WebAssembly, starting from the OCaml compiler lexer/parser, building my own type checker and immediate representation designed to target WebAssembly (which requires more type information than the OCaml compiler gives). I will then perform some optimizations (e.g. peephole optimizations, inlining of constants and functions), and translate into WASM, outputting a WebAssembly module and convenient JavaScript code to allow it to be easily integrated with other JavaScript code (WebAssembly does not have access to Web APIs on its own). As stretch goals I might expand the subset supported, implement more optimizations or implement garbage collection.

AFTER THE MEETING ALAN WANTS SUCCESS CRITERIA, EVAL CRIT AND TIMELINE

\section{Questions}

\subsection{OCaml Subset}
\begin{itemize}
\item What subset is useful and also interesting for the compiler?
\item What programs do we want to write for the web?
\item Do we extend OCaml to support JavaScript objects?
      \\ \textit{I think we can probably use a special data type?}
\end{itemize}

\subsection{Type Checker and Intermediate Representation}
\begin{itemize}
\item How do you write a type checker to keep required WASM type information?
      \\ \textit{Keep the type information in the intermediate representation, IR could have type data/annotations inferred by the type checker}
\item How to we design the intermediate representation to be suited for both OCaml and WebAssembly?
      \\ \textit{Type information, closures, control flow, remove pattern matching, remove polymorphism?}
\end{itemize}

\subsection{Optimizations}
\begin{itemize}
\item Should they be done to the immediate representation or to WASM output itself?
\item Which ones are both useful and interesting to implement?
\end{itemize}

\subsection{WASM}
\begin{itemize}
\item What kind of additional runtime system will be needed to support closures and complex values?
\item How will we use our code in the JavaScript environment?
\item How do we pass objects from the JS to our code?
\end{itemize}


\subsection{Stretch Goals / Garbage Collection}
\begin{itemize}
\item How do we output code in a way that makes this easier to implement? Put everything on the heap?
\item Can we do lifetime analysis in some simple cases?
\end{itemize}

\section{OCaml Features and how they map to WASM}
\subsection{Basics: Int, Bool, Fun, Closures}
\begin{minipage}{0.4\linewidth}
\begin{minted}{OCaml}
let myint = 12
let mybool = false
\end{minted}
\end{minipage}
\begin{minipage}{0.6\linewidth}
\begin{minted}{text}
valtype ::= i32 | i64 | f32 | f64
instr ::= (many obvious ones e.g. const, add)
\end{minted}
\end{minipage}
\\\\
WebAssembly only supports 32 and 64 bit ints: i32 and i64, so booleans will have to be encoded as ints as well. In OCaml ints are usually 63 bits, so i64 will be used.
\\\\
\begin{minipage}{0.4\linewidth}
\begin{minted}{OCaml}
let double x = 2 * x
let add x y = x + y
let add5 = add 5
\end{minted}
\end{minipage}
\begin{minipage}{0.6\linewidth}
\begin{minted}{text}
functype ::= [vec(valtype)] -> [vec(valtype)]
func ::= {type typeidx, locals vec(valtype), body expr}
expr ::= instr* end
\end{minted}
\end{minipage}
\\\\
WASM functions take a vector of valtypes (i32, i64, f32, f64), and return a vector that contains either 0 or 1 valtypes. As I will implement tuples later, this vector could be repurposed for curried arguments. But how will partial applications be dealt with? Perhaps with a closure that stores the curried arguments, we unload the arguments from the closure and call the main function? But how about when we need proper closures with values from the environment?
\\\\
Functions in WebAssembly have a type, a vector of locals, and then an expr as a body, where an expr is any number of instructions followed by an end marker.

\subsection{Recursion}
\begin{minipage}{0.4\linewidth}
\begin{minted}{OCaml}
let rec npower x n =
    if n = 0 then 1
    else x * npower x (n-1)
\end{minted}
\end{minipage}  
\begin{minipage}{0.6\linewidth}
\begin{minted}{text}
resulttype ::= void | valtype
instr ::= ... | if resulttype instr* end
              | if resulttype instr* else instr* end
\end{minted}
\end{minipage}
\\\\
Hopefully should be simple as calling the function from inside itself. Tail recursion elimination could be done as a potential optimisation.

\subsection{Pattern Matching}
\begin{minipage}{0.4\linewidth}
\begin{minted}{OCaml}
let iszero x =
    match x with
    | 0 -> true
    | _ -> false

(* Pattern matching variables *)
let hundredzero x =
    match x with
    | 0 -> 100
    | n -> n
\end{minted}
\end{minipage}  
\begin{minipage}{0.6\linewidth}
\begin{minted}{text}

\end{minted}
\end{minipage}
\\\\
It should be possible to transform the matching into a series of if/else statements (in the intermediate representation). Once you know which case you have, extracting the variables from the value should be fairly easy. This might be interesting for the type checker though, as for example inside a match branch we may have a better idea of the type of x than outside.

\subsection{Lists, Tuples and Types}
\begin{minipage}{0.5\linewidth}
\begin{minted}{OCaml}
type intlist = Nil
             | Cons of int * intlist
             
let rec sumlist = function
    | [] -> 0
    | (x::xs) -> x + sumlist xs
    
let rec append = function
    | ([], ys) -> ys
    | (x::xs, ys) -> x :: append (xs, ys)
\end{minted}
\end{minipage}  
\begin{minipage}{0.5\linewidth}
\begin{minted}{text}    
memarg ::= {offset u32, align u32}
instr ::= ...
        | inn.load memarg | inn.store memarg
        | inn.load8_sx memarg | inn.store8 memarg
\end{minted}
\end{minipage}
\\\\
These more complex values will need to be stored on a heap. WebAssembly provides memory areas, with size specified by page size (a page is 64kB). Thus we will need a runtime routine to allocate some memory for a complex value.
\\\\
A good way to represent these values might be with a tag e.g like an enum value (nil, cons), as well as a size, which would make things easier for garbage collection. We can then perform pattern matching by checking tags, and get inside values out by using a known offset as we know the type at compile time.
\\\\
The OCaml runtime has `unboxed ints' whereby a 64-bit value with MSB 0 is a pointer, while MSB 1 is an int (limiting ints to 63 bits). This requires extra arithmetic when dealing with ints, but avoids allocating `boxed ints' on the heap so e.g. if we have a list of ints, our Cons element doesn't have a pointer to the int, it has the int.
\\\\
A possible optimisation with list constants \camlinline{let mylist = [1;2;3;]} would be to allocate all this memory in a continuous block.

\subsection{Polymorphism}
\begin{minipage}{0.4\linewidth}
\begin{minted}{OCaml}
type 'a list = Nil
             | Cons of 'a * 'a list

let rec map f = function
    | [] -> []
    | (x::xs) -> (f x) :: (map f xs)
\end{minted}
\end{minipage}  
\begin{minipage}{0.6\linewidth}
\begin{minted}{text}

\end{minted}
\end{minipage}
\\\\
I thought that one way of doing this might be like C++ templates: whenever we have a polymorphic function call with a specific type, we have a different version of that function in the final output for each type that it is used with. Likewise we could treat the final types differently. This gives the restriction that these functions can't be called from the JavaScript environment, however it's likely that the only functions that will be able to be called will be ones that accept `JavaScript objects' as parameters.
\\\\
Another way might be like Java: as all our points will be 64-bit integers, it doesn't matter which type we are passing in, except with `primitives' e.g. 32-bit integers for booleans, or floats/doubles.

\subsection{Exceptions}    
\begin{minipage}{0.4\linewidth}
\begin{minted}{OCaml}
exception Error of int
let error_not_zero = function
    | 0 -> 0
    | x -> raise Error x
    
let myerrorednum =
    try 
        error_not_zero 86
    with Error x ->
        x
\end{minted}
\end{minipage}  
\begin{minipage}{0.6\linewidth}
\begin{minted}{text}

\end{minted}
\end{minipage}
\\\\
Not yet supported by WebAssembly.
\\\\
https://github.com/WebAssembly/exception-handling/blob/master/proposals/Exceptions.md
\\\\
Could possibly implement by making functions that can throw exceptions return a special type, which would add a large overhead. But then any function that takes a function as an argument could throw an exception. Maybe the type system could deal with that though by compiling a separate version of e.g. map that takes functions that throw exceptions as an argument, and one that does not. Possible stretch goal?

\subsection{Functions as Arguments / Anonymous Functions}
\begin{minipage}{0.4\linewidth}
\begin{minted}{OCaml}
let rec map f = function
    | [] -> []
    | (x::xs) -> (f x) :: (map f xs)
let mult3 = map (fun x -> 3 * x) [1;2;3]
\end{minted}
\end{minipage}  
\begin{minipage}{0.6\linewidth}
\begin{minted}{text}

\end{minted}
\end{minipage}
\\\\
Anonymous functions should hopefully be as simple as extracting the anonymous function and giving it a generated name, and then using that name.
\\\\
Functions as arguments will need closures, which contain a function pointer and an environment. How will closures get on with the WASM type system?

\subsection{References and Loops}
\begin{minipage}{0.4\linewidth}
\begin{minted}{OCaml}
(* References *)
let myref = (ref 5);;
myref := 10;;
let refval = !myref;;

(* Loops *)
for i = 0 to 10 do
    myref := !myref + i
done;;

while (!myref < 1000) do
    myref := !myref * 2
done;;
\end{minted}
\end{minipage}  
\begin{minipage}{0.6\linewidth}
\begin{minted}{text}
instr ::= ...
        | loop resulttype instr* end
        | br labelidx
        | br_if labelidx
\end{minted}
\end{minipage}
\\\\
References can be special types stored on the heap. Should just use the same memory operations as in the types section, but now we are modifying existing values.
\\\\
When we enter a loop (or a block) in WebAssembly, a label is pushed to the stack. The label at the top of the stack is 0, then 1, etc... So \textinline{br 0} would branch to the most recent label. \textinline{br_if} will pop an i32 from the top of the stack, and execute if that is non-zero.

%\chapter{Proposal}

\chapter{Project Setup}

\section{Repository Overview}
\begin{itemize}
	\item \textbf{documents}: Holds important project documents such as this
	\item \textbf{demo}: Demo project structure for experimenting with Compiler Libs and Dune
\end{itemize}

\section{Build System}
Dune\footnote{https://github.com/ocaml/dune} is the build system I will be using. This should be able to handle building the compiler and running the tests without need for another tool (although I could add a Makefile to invoke dune).
\\\\
In order to use the compiler libs in Dune, the following can be used
\begin{minted}{lisp}
(executable
	(name my_ml_file)
	(libraries compiler-libs.common))
\end{minted}
The book \textit{Real World OCaml} recommends using \textit{Base} and \textit{Stdio} as replacements for the usual OCaml standard library, so I will be using those.
\\\\
Actually, it's a lot more complicated than that. At the top of the project, a \textinline{proj.opam} file is needed, although I have no idea what needs to go in this yet (mine is just the opam version), I think it's for specifying the overall build and test of the project. A \textinline{dune} file is required in every directory. At the top-level, so in one directory, we will have an executable like above, but the others will need to be libraries, like so
\begin{minted}{lisp}
(library
    (name proj_inc_nums)
    (public_name proj.inc_nums)
    (libraries core_kernel compiler-libs.common))
\end{minted}
If this library is included in another library/executable, you use the public name \textinline{proj.inc_nums}, which allows you to use \textinline{Proj_inc_nums} to access all modules in the library. Thus using \camlinline{open Proj_inc_nums} is probably a good idea. The library includes all files in the directory as modules. A file called \textinline{inc_nums.ml} will be the OCaml module \textinline{Inc_nums}.
\\\\
It will probably be a good idea to add a \text{.mli} file for each ml file to provide a signature and hide the implementation.

\section{Parsing}
The module \camlinline{Parse} handles parsing from strings to ASTs. It has several different functions depending on what you want to parse. \camlinline{toplevel_phrase} includes things that might be typed into toplevel, which are either structures or directives (e.g. \camlinline{#require "core"}). \camlinline{implementation} is what I will use to parse structures, ignoring these directives.

\section{Type Checking}
Mostly based on the article http://dev.stephendiehl.com/fun/006\_hindley\_milner.html\#observations. Using Hindley Milner
\\\\
What's the difference between a \textbf{Scheme} and a \textbf{Scheme Type}? A scheme type is a type e.g. with vars and stuff, while a scheme itself is a forall over a bunch of type variables, followed by a type. Thus we can instantiate a scheme to a scheme type by substituting the variables in the set for fresh variables, and likewise we can generalize a scheme type to a scheme by finding all the type variables unique to it (not in the context), and making a Forall over those variables.
\\\\
HM is all about \textbf{let-polymorphism}. When we have \camlinline{let x = e1 in e2}, we generalize the resulting type of x, and every time we use x in e2, we create a fresh instantiation. Thus it will typecheck differently to \camlinline{(fun x -> e2) (e1)}.
\\\\
Actually in OCaml there are two types of let-binding: the non-recursive one and the recursive one. These are type-checked differently:
\begin{itemize}
	\item Non-recursive: We can use previous variable names, but not the current one or future ones
	\item Recursive: We can use all the variable names in the binding
\end{itemize}

\subsection{Issues}
\begin{itemize}
	\item (Solved) MAJOR: Once you use a polymorphic function, it gets bound to those types. Need to implement generalization and instantiation. \camlinline{let id x = x in (id 1, id true)} Possible solutions:
	\begin{itemize}
		\item A forall type. I like this solution better. After we make a function, all unique type variables in it are forall.
		\item Schemes. Forall type but explicit.
		\item Functions are implicitly forall? Then we could mark type variables as weak or not. A weak type variable cannot be unified to be a strong one, but a strong one can be unified to be weak. Instantiating functions equates to replacing all their strong type variables with fresh new strong ones. So what is weak and what is strong:
		\begin{itemize}
			\item Function argument: strong
			\item Let x =: weak
		\end{itemize}
	\end{itemize}
    \item (Solved) MAJOR: Occurs check. \camlinline{let f x = x x in f} (done!)
    \item (Stretch Goal) MAJOR: Error messages and unification. Can this be done with unification at the end? Should unification be done at the end like it is?
    \item MAJOR: Additional checks. Plenty of things typecheck that shouldn't
    \begin{itemize}
    	\item (Solved) Check that type constructors use valid types as arguments (done!)
    	\item (Solved) Occurs check
    	\item (Solved) Check that constraints use valid types (sounds like we need a function which takes a context and a type and checks if it is valid) (done!)
    	\item (Will come back to if/when it affects code generation) Let rec restrictions, e.g. prevent cycles
    \end{itemize}
\end{itemize}

\subsection{Still Needs Doing}
\begin{itemize}
	\item (Done) Various other expression types e.g. if, match
	\item (Done) Let binding structures
	\item (Done) Type structures and custom data types
	\item Strong a tree of the partial type data. We would then do a substitution pass over this at the end with the final substitution.
	\item (Stretch Goal) Decent error messages i.e. not `UnificationFailure'
	\item Write some Unit Tests. Potentially would be more possible when I know the final signatures
\end{itemize}

\section{Intermediate Representation}
\subsection{Closure Conversion}
Many programming languages only allow top-level functions, but of course OCaml allows nested functions, and anonymous functions. We also need to deal with closures. This is fairly standard except recursive closures contain a reference to themselves.

\subsection{Match Statements}
We use the block feature of the IR, with one large block to enable escaping from the entire match without checking additional cases, and individual blocks for each case so that we can jump out of a case as soon as we don't match it and move on to the next case.

\subsection{Recursive Closures}
This is probably the trickiest case, but here's what we (will) do
\begin{itemize}
	\item Get a list of the function names
	\item Make named vars for each of these
	\item Make empty closures for each of these, and assign that to the var
	\item Fill these closures one by one
\end{itemize}

REMEMBER YOU DIDN'T FINISH SOME CASES LOL

\subsection{Success Criterion and changes}

\begin{itemize}
	\item Use i32 instead of i64: JavaScript doesn't support getting/setting 64-bit WA values
\end{itemize}

\subsection{Exporting Functions}
At the moment the global variable is the closure. Two options:
\begin{itemize}
	\item Provide a way to call closures from JS, would require extra JS API
	\item Make global variable function, with a weird name for the closure. Probably needs more code, but would make exported WebAssembly module much cleaner and more portable / easier to use
\end{itemize}

\subsection{Stack Based IR Code Length}
\begin{minted}{text}
67 samples/arithmetic.wast
273 samples/basic.wast
132 samples/closure_basic.wast
420 samples/construct.wast
756 samples/core_subset.wast
76 samples/float_construct.wast
142 samples/floats.wast
148 samples/for.wast
507 samples/functions.wast
454 samples/match.wast
426 samples/poly-advanced.wast
263 samples/poly-basic.wast
728 samples/poly-construct.wast
119 samples/rec.wast
142 samples/refs.wast
67 samples/tuple.wast
41 samples/unop.wast
132 samples/while.wast
4893 total
\end{minted}

\end{document}