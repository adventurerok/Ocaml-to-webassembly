% The master copy of this demo dissertation is held on my filespace
% on the cl file serve (/homes/mr/teaching/demodissert/)

% Last updated by MR on 2 August 2001

\documentclass[12pt,twoside,notitlepage]{report}

\usepackage{a4}
\usepackage{verbatim}

\usepackage{amsmath}
\usepackage{amssymb}
\usepackage{mathtools}

\usepackage{dirtree}

\usepackage{minted}
%\usemintedstyle{colorful}
\setmintedinline{breaklines}

\newcommand{\textinline}{\mintinline{text}}
\newcommand{\cinline}{\mintinline{C}}
\newcommand{\camlinline}{\mintinline{OCaml}}
\newcommand{\wainline}{\mintinline{LISP}}

\newcommand\note[1]{\textcolor{blue}{#1}}

\input{epsf}                            % to allow postscript inclusions
% On thor and CUS read top of file:
%     /opt/TeX/lib/texmf/tex/dvips/epsf.sty
% On CL machines read:
%     /usr/lib/tex/macros/dvips/epsf.tex



\raggedbottom                           % try to avoid widows and orphans
\sloppy
\clubpenalty1000%
\widowpenalty1000%

\addtolength{\oddsidemargin}{6mm}       % adjust margins
\addtolength{\evensidemargin}{-8mm}

\renewcommand{\baselinestretch}{1.1}    % adjust line spacing to make
                                        % more readable

\usepackage[backend=bibtex, style=alphabetic, sorting=ynt]{biblatex}
\addbibresource{refs.bib}

\begin{document}



%%%%%%%%%%%%%%%%%%%%%%%%%%%%%%%%%%%%%%%%%%%%%%%%%%%%%%%%%%%%%%%%%%%%%%%%
% Title


\pagestyle{empty}

\hfill{\LARGE \bf Paul Durbaba}

\vspace*{60mm}
\begin{center}
\Huge
{\bf Compiling OCaml to WebAssembly} \\
\vspace*{5mm}
Diploma in Computer Science \\
\vspace*{5mm}
Robinson College \\
\vspace*{5mm}
May 2020  % today's date
\end{center}

\clearpage


 
\newpage
\section*{Declaration}

I, Paul Durbaba of Robinson College, being a candidate for Part II of the Computer
Science Tripos, hereby declare
that this dissertation and the work described in it are my own work,
unaided except as may be specified below, and that the dissertation
does not contain material that has already been used to any substantial
extent for a comparable purpose.

\bigskip
\leftline{Signed Paul Durbaba}

\medskip
\leftline{Date [date]}

\section*{Acknowledgements}

% TODO List the people that check the diss
\note{LIST THE PEOPLE THAT CHECK THE DISS}

%%%%%%%%%%%%%%%%%%%%%%%%%%%%%%%%%%%%%%%%%%%%%%%%%%%%%%%%%%%%%%%%%%%%%%%%%%%%%%
% Proforma, table of contents and list of figures

\setcounter{page}{1}
\pagenumbering{roman}
\pagestyle{plain}

\chapter*{Proforma}

{\large
	\begin{tabular}{ll}
		Name:               & \bf Paul Durbaba                       \\
		College:            & \bf Robinson College                     \\
		Project Title:      & \bf Compiling OCaml to WebAssembly \\
		Examination:        & \bf Part II Computer Science, May 2020        \\
		Word Count:         & \bf FILL IN LATER  \\
		Project Originator: & Timothy M. Jones                \\
		Supervisor:         & Tobias Kohn            \\ 
	\end{tabular}
}
%\footnotetext[1]{This word count was computed
%	by {\tt detex diss.tex | tr -cd '0-9A-Za-z $\tt\backslash$n' | wc -w}
%}
%\stepcounter{footnote}


\section*{Original Aims of the Project}

\note{At most 100 words describing the original aims of the project.}


\section*{Work Completed}

\note{At most 100 words summarising the work completed.}

\section*{Special Difficulties}

\note{At most 100 words describing any special difficulties that you faced.
(In most cases the special difficulties entry will say “None”.) }

\tableofcontents

\listoffigures

%%%%%%%%%%%%%%%%%%%%%%%%%%%%%%%%%%%%%%%%%%%%%%%%%%%%%%%%%%%%%%%%%%%%%%%
% now for the chapters

\clearpage        % just to make sure before the page numbering
                        % is changed

\setcounter{page}{1}
\pagenumbering{arabic}
\pagestyle{headings}

\chapter{Introduction}
\note{The Introduction should explain the principal motivation for the project. Show how the work fits into the broad area of surrounding Computer Science and give a brief survey of previous related work. It should generally be unnecessary to quote at length from technical papers or textbooks. If a simple bibliographic reference is insufficient, consign any lengthy quotation to an appendix.}
%TODO EXPLAIN KEY MOTIVATION, IDEAS BEHIND PROJECT
My motivation for this project is to learn how to make a compiler. Compilers are essential to computing because they enable the transformation of code from high-level languages that are intuitive for use by us, into low-level machine understandable code than can actually be executed, and writing a compiler is a substantial software project that invokes many areas of computer science such as type theory and program analysis.

% TODO PREVIOUS RELATED WORK

\section{OCaml}
OCaml\cite{OCaml} is a strongly-typed functional programming language, with some imperative features such as references. I chose OCaml both as the source language of the compiler, and the language the compiler is designed in, because I wanted to gain some familiarity in writing programs in functional languages, and OCaml has similar syntax to Standard ML taught in first year, but with much better library support, and because compiling a functional programming language presents additional challenges to compiling an imperative language such as C - with first class functions and pattern matching requiring special consideration.

\section{WebAssembly}
% copied from project proposal
WebAssembly\cite{webassembly} is a stack-based binary instruction format for the web. It is not a replacement of JavaScript, instead working alongside JavaScript with the main goal of improving the performance of  more computationally intensive functions in web applications, leaving tasks like DOM manipulation to JavaScript (there are no plans for WebAssembly to support this) It  is  expected  that  a  JavaScript  application might  call  some  functions  implemented  in  WebAssembly  to  perform  computation,  and then display the results itself.
\\\\
The current MVP (minimum viable product) version of WebAssembly is designed for compiling languages like C and C++ that do not use garbage collection and can make do without exceptions. There are extensions currently being developed to add support for these features, but progress on these is slow as they often depend on other extensions, for instance the garbage collection extension depends on extensions for reference types and typed function references, which seek to expand the WebAssembly type system so for instance a garbage collector would understand the shape of data in memory\cite{Wgce}.
\\\\
WebAssembly was chosen as the target instruction set because it is relatively new, with few compilers out there currently targeting it, and it is likely to grow in popularity in the future as more extensions are added to it that make it more viable to be used - such as support for garbage collection and exceptions.

\note{TODO Alan suggests moving to Preparation, and giving an overview of WA's linguistic features. I already have a bit of an overview in the code generation section, but these are the main ones:
\begin{itemize}
	\item Stack based, structured (but compiler doesn't use this structure much / goes for unstructured options)
	\item Local variables get/set/tee
	\item No swap or dup instructions
	\item Verified i32/f32 types, must know signature of function for all function calls, blocks have a result type
	\item Can branch only out of a block early, or to the start of a loop
	\item JS interface can only pass i32 and f32 values - no strings / structs or anything
\end{itemize}}
% TODO

%TODO WHAT IS WEBASSEMBLY


% TODO DESCRIPTION OF HOW TO BUILD THE PROJECT?

\section{Related Work}
% TODO Js\_of\_ocaml, A

There have been a few attempts to compile OCaml to WebAssembly already, such as by @SanderSpies\cite{Awbfo}, who modified the existing backend of the OCaml Compiler to target WebAssembly. Their attempt worked from the `CMM' --- the final stage of the OCaml compiler before code generation, modifying that to include extra type information for WebAssembly, and then doing code generation. This differs from my approach in that I have implemented almost an entire compiler from the type-checker through to the WebAssembly code generator, but excluding lexing/parsing.
\\\\
While Sander's approach allows them to leverage the existing features and optimisations of the OCaml compiler, their approach didn't fit with my goals of learning how to write an entire compiler - such an approach wouldn't give me the freedom and opportunity to explore aspects of compiler construction such as type-checking, intermediate representations, and optimisations, which are some of the most interesting parts of the project.


\clearpage



\chapter{Preparation}
\note{Principally, this chapter should describe the work which was undertaken before code was written, hardware built or theories worked on. It should show how the project proposal was further refined and clarified, so that the Implementation stage could go smoothly rather than by trial and error.}

\note{Throughout this chapter and indeed the whole dissertation, it is essential to demonstrate that a proper professional approach was employed.}

\note{The nature of this chapter will vary greatly from one dissertation to another but, underlining the professional approach, this chapter will very likely include a section headed “Requirements Analysis” and incorporate other references to software engineering techniques.}

\note{The chapter will cite any new programming languages and systems which had to be learnt and will mention complicated theories or algorithms which required understanding.}

\note{It is essential to declare the Starting Point (see Section 7). This states any existing codebase or materials that your project builds on. The text here can commonly be identical to the text in your proposal, but it may enlarge on it or report variations. For instance, the true starting point may have turned out to be different from that declared in the proposal and such discrepancies must be explained. }

\section{Requirements}

The success criteria in the project proposal presents a clearly defined subset of OCaml to implement. This subset was designed to be large enough so that useful OCaml programs could be written in it, while small enough to be feasible to implement by Christmas.

% TODO STARTING POINT

% TODO MATERIAL DONE BEFORE CODE WAS WRITTEN

% TODO? HOW I ENSURED CODING WASN'T TRIAL AND ERROR

\section{Components of the Compiler}
\note{Write after implementation}

\section{Working Environment and Tools Setup}
Since I have both a laptop and desktop, I decided the best way to work on both would be to do the work remotely via a remote desktop application on a remote server. Both my laptop and desktop were configured to download backups from the server once per hour (if they were on), so if there proved to be a problem with the server, I could redeploy easily by uploading the backups to a new server if required.
\\\\
In addition, I used Git in order to keep a record of my work, to allow me to access previous versions of files, and to backup to GitHub

\subsection{Dune}
I chose to use the Dune\cite{Dune} build system for OCaml as it is the most widely-used build system for OCaml, and supports multi-module projects and dependencies installed via OPAM, the OCaml Package Manager.
\\\\
Dune build files are specified in each directory in a file called \textinline{dune}, with the top-level directory specifying the build file for the entire project, and subdirectories containing the build files for each module. These build files are specified in a LISP-like `s-expression' syntax, for instance here is the top-level build file of the project:
\\\\
\begin{minipage}{\linewidth}

\begin{minted}{LISP}
(executable
    (name toplevel)
    (libraries proj.types proj.transform proj.codegen proj.base
               core_kernel compiler-libs.common)
    (preprocess (pps ppx_jane)))
\end{minted}
\end{minipage}
\\\\
The name field includes the `public names' of libraries to be included. Library modules have their own build files that specify `library' instead of `executable', and an additional `public\_name' field.
\\\\
The executable can be build by invoking \textinline{dune build toplevel.exe} in the top-level directory, which will output the binary to \textinline{_build/default/toplevel.exe}.

\note{Easy to shorten?}

\section{Libraries / Tools Used}
\subsection{OCaml Compiler Libs}
I used the official OCaml compiler \cite{OCaml} libraries to perform lexing and parsing, which provide the same frontend as used by the official OCaml compiler. As the frontend of the official OCaml compiler is liable to change between releases, I stuck to version 4.08, which was the most recent version when I started the project but has now been succeeded by 4.09.

\subsection{WebAssembly Binary Toolkit}
The WebAssembly Binary Toolkit \cite{Wabt} (WABT)  is a separate tool which can compile WebAssembly Text Form (.wast files) outputted by my compiler to the WebAssembly Binary Format (.wasm files) that can be loaded by NodeJS and browsers.
\\\\
\note{TODO Explain how to use compiler, then invoke WABT, and then explain JS code to load the WASM? The E2E tester does this so could also be explained in Evaluation chapter?}

\section{Starting Point}
I had very little OCaml experience prior to this project, and experience of writing compilers. In addition, I had no experience of using WebAssembly, but I have plenty of experience writing JavaScript.
\\\\
I attempted to deal with some of these issues prior to starting the project by coming up with some OCaml samples for the compiler to compile in the future, and by setting up an OCaml workspace where I successfully figured out how to import the OCaml Compiler libraries, and learned to navigate the AST they use by writing a simple example that adds one to integer constants. I also read through the WebAssembly documentation to get a sense of which features would be a challenge to compile to WebAssembly.

\clearpage
\chapter{Implementation}
%\note{This chapter should describe what was actually produced: the programs which were written, the hardware which was built or the theory which was developed. Any design strategies that looked ahead to the testing stage might profitably be referred to (the professional approach again).}
%
%\note{Descriptions of programs may include fragments of high-level code but large chunks of code are usually best left to appendices or omitted altogether. Analogous advice applies to circuit diagrams.}
%
%\note{Draw attention to the parts of the work which are not your own. The Implementation Chapter should include a section labelled "Repository Overview". The repository overview should be around one page in length and should describe the high-level structure of the source code found in your source code Repository. It should describe whether the code was written from scratch or if it built on an existing project or tutorial. Making effective use of powerful tools and pre-existing code is often laudable, and will count to your credit if properly reported.}
%
%\note{It should not be necessary to give a day-by-day account of the progress of the work but major milestones may sometimes be highlighted with advantage. }

% TODO WORK DONE, ONE SECTION PER PART OF THE COMPILER



\section{Front End}
Lexing and parsing is handled by the OCaml Compiler Libs, which produces a \textinline{structure} object representing the AST of the entire program. This AST and the corresponding typed-AST, which I designed to be very similar (using the same names prefixed with a t), are used throughout the first half of the compilation process before intermediate translation occurs. Figure \ref{fig:ast} gives the grammar of the OCaml AST in Extended Backus-Naur form \cite{Ebnf}, with an overview of it's components following. The typed-AST holds additional type information and prefixes its elements with `t'.
\begin{figure}[h!]
	\begin{minted}{EBNF}
structure = { structure_item } ;
structure_item = "expression" expression
               | "let" rec_flag { value_binding }
               | "type" rec_flag { type_decl } ;

type_decl = name { type_param } { constructor_decl } ;
constructor_decl = name { type } ;
type = "int" | "float" | "bool" | "unit" | type_param | type "->" typ
     | "(" { type } ")" | name "(" { type } ")" ;
type_param = "`" name ;

expression = "constant" constant | "identifier" identifier
           | "fun" pattern expression
           | "app" expression { expression }
           | "let" rec_flag { value_binding } expression
           | "tuple" { expression } | "construct" name { expression }
           | "if" expression expression [ expression ]
           | "match" expression { case }
           | "constraint" expression type ;
           
pattern = "any" | "identifier" identifier | "constant" constant
        | "tuple" { pattern } | "construct" name { pattern }
        | "constraint" pattern type ;
        
case = pattern expression ;
value_binding = pattern expression ;
           
identifier = predefined | name ;
predefined = "+" | "-" (* All predefined operators *) ;
constant = (* A string representing a constant, e.g. "34", "()", "false" *) ;
name = (* A string *)
	\end{minted}
	\caption{An grammar for the OCaml Abstract Syntax Tree}
	\label{fig:ast}
\end{figure}
\begin{itemize}
	\item \textinline{(t)structure} A list of \textinline{(t)structure_item} representing the top-level items in a file.
	\item \textinline{(t)structure_item} A single top-level item in a file. Can either be an \textinline{(t)expression} (e.g. for imperative code), a let binding consisting of a list of \textinline{(t)value_binding}, or a type definition consisting of a list of \textinline{type_decls}
	\item \textinline{(t)expression} A unit of code that will produce a value. There are various types including identifiers, constants, match statements, function definitions and applications.
	\item \textinline{(t)pattern} A structure of a value, that can be used both to test if a value conforms to this structure (e.g. in a match statement), and to destructure a value into separate variables (e.g. in a let binding). Can be an identifer, constant, tuple of several patterns, or a construct (effectively a named tuple).
	\item \textinline{(t)value_binding} An expression and a pattern. The result of evaluating the expression is destructured and bound to the variables in the pattern, for instance in \camlinline{let x,y = (5 + 2, 3)} the pattern is \camlinline{(x, y)} and \camlinline{(5 + 2, 3)} is the expression
\end{itemize}

\note{Easy to shorten?}
\note{Put the AST in an appendix? Get rid of the explanation?}





\section{Type Checker}
%\note{
%	Does multiple things
%	\begin{itemize}
%		\item HM type inference, generating constraints and then unifying them
%		\item Builds up information about type definitions e.g. construct types and their constructors
%		\item Makes the typed-ast
%		\begin{itemize}
%			\item Small differences to make the typed-ast nicer to work with
%			\item Variables get additional unique ID
%			\item Patterns get a list of variables they define and their types
%			\item Value bindings get a similar list with the generalized types
%		\end{itemize}
%	\end{itemize}
%}
The type-checker is responsible for translating the untyped AST into a typed-AST, and in doing so ensuring that the program is well typed, inferring types that are unknown. The main algorithm that I use is a inference algorithm based on the Hindley-Milner type system. This is a constraint based algorithm that builds up a set of constraints about types at different points in the program, and solves them to produce a unification mapping type-variables to types. My implementation was guided by an article about using Hindley-Milner in Haskell\cite{Hmi}.
\begin{figure}[h]
	$$\begin{array}{clr}
	\dfrac{x : \sigma \in \Gamma \quad \tau =\ \text{instantiate}(\Gamma, \sigma)}{\Gamma \vdash x : \tau} && \textsc{Var} \\\\
	\dfrac{\Gamma \vdash e_0 : \tau \rightarrow \tau' \quad \Gamma \vdash e_1 : \tau}{\Gamma \vdash e_0\ e_1 : \tau'} && \textsc{App} \\\\
	\dfrac{\Gamma, x : \tau \vdash e : \tau'}{\Gamma \vdash \lambda x . e : \tau \rightarrow \tau'} && \textsc{Abs} \\\\
	\dfrac{\Gamma \vdash e_0 : \tau \quad \Gamma, x : \text{generalize}(\Gamma, \tau) \vdash e_1 : \tau'}{\Gamma \vdash \mathtt{let}\ x = e_0\ \mathtt{in}\ e_1 : \tau'} && \textsc{Let} \\\\
	\text{generalize}(\Gamma, \tau) = \forall \hat{\alpha} . \tau \quad \text{where}\ \hat{\alpha} = \text{free}(\tau) - \text{free}(\Gamma) && \\
	\text{instantiate}(\Gamma, \forall \hat{\alpha} . \tau) = [\hat{\beta} / \hat{\alpha}]\tau \quad \text{where}\ \hat{\beta} \cup \text{free}(\Gamma) = \emptyset
	
	\end{array}$$
	\caption{The basic rules of a HM type system, taken from the Wikipedia page\cite{Hmts}}
	\label{fig:hm}
\end{figure}
\\\\
Figure \ref{fig:hm} shows the rules for a basic Hindley-Milner type system. Types can either be atomic types (such as e.g. \textinline{int}), type variables or function types of the form $\tau \rightarrow \tau'$. For a rule like \textsc{Abs}, it is unclear what type we should chose for $\tau$. In such a case where we need a type and it's not available at that point, the Hindley-Milner type inference system introduces a fresh (previously unused) type variable to represent that type. We then modify the rules so that instead of checking types as they go, they produce constraints that can be solved later. Solving these constraints produces a substitution mapping type variables to types, that can then be substituted into all of the types in the AST to eliminate variables where we now know the type.
\\\\
Suppose we have a variable \textinline{id} which refers to the identity function \camlinline{fun x -> x} of type $\beta \rightarrow \beta$. We want to be able to use this function on different types, e.g. \camlinline{id 3} and \camlinline{id true}. This requires generalization and instantiation. A generalized type is represented as $\forall \hat{\alpha}. \tau$, where $\hat{\alpha}$ is a set of variables which could take on multiple types, e.g. the identity function is of type $\forall \beta. \beta \rightarrow \beta$. A type can be generalized by extracting all free variables it contains, minus those from the context: if the type-variable is also used elsewhere, it might eventually become a concrete type. Instantiation of a generalized type is simply replacing each generalized type-variable in the type with a fresh type-variable, so one use of \camlinline{id} might type it as $\delta \rightarrow \delta$ while another $\epsilon \rightarrow \epsilon$, allowing both to be separately unified with a boolean and an integer type for instance.
\\\\
In a case like \camlinline{(fun y -> (y 3, y true)) (fun x -> x)}, one might think this would type-check by generalizing the type of the identity function. However, type-checking such expressions correctly, such as in the polymorphic lambda calculus (System F) is undecidable in general \cite{SystemFUndecidable}. Instead, my compiler implements let-polymorphism, where generalization is restricted to variables in let bindings, which is also reflected in the rules in figure \ref{fig:hm}.
\\\\
My implementation uses the types defined in Figure \ref{fig:types}. A \camlinline{scheme} is a generalized type. Type constraints are implemented as \camlinline{scheme_type} pairs.
\begin{figure}[h]
	\begin{minted}[linenos]{OCaml}
	type tvalue = V_unit | V_int | V_bool | V_float
	
	type scheme_type =
	| T_var of string (* Type variable *)
	| T_val of tvalue
	| T_tuple of scheme_type list
	| T_constr of string * scheme_type list (* Constructs *)
	| T_func of scheme_type * scheme_type (* Function type *)
	
	type scheme = Forall of String.Set.t * scheme_type
	\end{minted}
	\caption{The definition of types}
	\label{fig:types}
\end{figure}
\\\\
A context object to keep track of the mapping between variable names and their generalized \camlinline{scheme_type}, as well as a unique ID for that variable --- variable names are not unique in the OCaml AST, but depend on the local scope, a problem that can be fixed in the typed-AST by including the unique ID in the variable's identifier. In addition, this context keeps track of custom types added and the constructors for those types --- for instance a tree type might be defined with leaf and node constructors each taking different arguments.
\\\\
There are four main routines used in the type-checker, each of which are mutually recursive with each other:
\begin{itemize}
	\item \camlinline{infer_expr} performs type inference on an expression, outputting the typed-AST expression and a list of constraints. Typing an expression is relatively simple:
	\begin{enumerate}
		\item Call \camlinline{infer_expr} recursively on sub-expressions
		\item Introduce a fresh type-variable $\alpha$ to represent the type of the entire expression
		\item Add constraints based on this and the types of sub-expressions
		\item Build typed-AST from sub-expressions' typed-ASTs, and tag it with type $\alpha$
		\item Output this typed-AST and all the constraints (including those from sub-expressions)
	\end{enumerate}
	As an example, take a function application $e_1\ e_2$, where $e_1 : t_1$ and $e_2 : t_2$. We introduce the fresh type-variable $\alpha$ as the type of the entire expression, and then add the constraint $t_1 = t_2 \rightarrow \alpha$. The typed-AST form of the expression we output would be $((e_1 : t_1)\ (e_2 : t_2)) : \alpha$.
	
	\item \camlinline{infer_pattern} provides a typed-AST representation of a pattern. This representation includes both the overall type of the pattern, and a mapping of variables contained within the pattern to types. This algorithm has no general-form, but has cases depending on the type of pattern:
	\begin{itemize}
		\item Variable patterns introduce a fresh type-variable for the variable, and add this variable to the variable mapping
		\item Constant patterns simply output the type of the constant
		\item Tuple and construct patterns recursively call \camlinline{infer_pattern} on their sub-patterns, and then build a tuple or construct type from the resulting types, merging together the variable mappings.
	\end{itemize}

	\item \camlinline{type_expr} Is shorthand for inferring types on an expression and solving the constraints. It still outputs `outer constraints' however --- constraints on type-variables that occur in the context passed into \camlinline{type_expr}.
	
	\item \camlinline{ctx_of_bindings} Is responsible for type-checking a list of value-bindings (let bindings), outputting both a new context containing the new bound variables and the typed-AST representation of the value-bindings. It must generalize the types of the new variables, and handle both recursive and non-recursive cases. The recursive case proceeds as follows:
	\begin{enumerate}
		\item Extract all the patterns from the bindings, and type-check those. This gives us the non-generalized type of each variable, which is used to create a temporary context
		\item Infer types on each binding's expression, building up a set of constraints (of which unifying the pattern type and expression type for each binding are included).
		\item Solve those constraints and substitute into each binding's type. We can now generalize each variable's type, avoiding type-variables that are also free in the original context.
		\item These generalized variable types give us the final context to output, outputting in addition the typed-AST bindings and the outer constraints extracted from the constraint set generated earlier.
	\end{enumerate}
	The non-recursive case is a simplification of this where each binding is treated separately, and no temporary context is needed.
\end{itemize}

The overall algorithm takes an untyped-AST, and outputs a typed-AST and a context containing user defined types and their constructors, which can be passed through to the lambda-lifting and closure-conversion stage.

\section{Lambda Lifting / Closure Conversion}
%\note{Replace function definitions with mk\_closure, extracting function to it's own object. Track the free variables inside the function}

Lambda Lifting and Closure Conversion is the process whereby function definitions are extracted from the AST, replacing the original definition site with a special operation that constructs a closure of the original function, passing in the required environment variables as needed.
\begin{figure}[h]
	\begin{minted}[linenos]{OCaml}
(* Curried function definition *)
let sum (x, y) (z, w) = x + y + z + w

(* Curried function representation in AST *)
let sum = (fun (x, y) -> (fun (z, w) -> x + y + z + w))

(* Modified AST and extracted functions *)
let sum = mk_closure $$f_sum ()
$$f_sum (): fun arg_sum -> mk_closure $$f_sum-app (arg_sum)
$$f_sum-app (arg_sum): fun arg_sum-app ->
let (x, y) = arg_sum in
let (z, w) = arg_sum-app in
x + y + z + w
	\end{minted}
	\caption{Curried functions in closure conversion}
	\label{fig:curried}
\end{figure}
\\\\
This is achieved by walking the typed-AST. Each time a function definition is encountered, we must construct an extracted function definition, and modify the AST to include a make-closure operation. This is done in the following order, with figure \ref{fig:curried} used to illustrate:
\begin{enumerate}
	\item We give a unique name the function. There are three possible cases here:
	\begin{itemize}
		\item The function is defined in a let binding (e.g. \camlinline{sum}), in which case we use the name of the variable in the let binding.
		\item The function is defined as the expression of another function. This occurs in curried functions which are represented in the AST as nested function definitions. In this case we take the name of the parent function, and append `-app'. (e.g. \camlinline{sum-app})
		\item The function is defined anonymously, in which case we give it an anonymous name.
	\end{itemize}
	\item For curried functions, we create a new argument (single variable pattern) for each function (e.g. \camlinline{arg-sum}), and put let expressions to bind these arguments to the original patterns inside the body of the innermost function in the curried definition. This ensures that we only need to store one environment variable per curried argument inside the closures of deeper functions. For instance, \camlinline{arg-sum} is stored inside the closure for \camlinline{$$f_sum-app}, instead of needing to store both \camlinline{x} and \camlinline{y}.
	
	\item We recursively apply closure conversion to the expression of the function, to give us the expression to use in the extracted function definition.
	\item We perform a depth-first search of this expression to determine the free variables, which will become `closure arguments' for the extracted function. For instance, \camlinline{arg-sum} is free inside the body of \camlinline{$$f_sum-app}, so it must become a closure argument for this extracted function.
	\item We replace the function definition with a `special' make closure operation, which takes the name of the extracted function and the closure arguments as parameters. \camlinline{mkclosure $$f_sum-app (arg_sum)}
	\item We create an object to represent the extracted function, which contains it's name, argument pattern, expression, and list of closure variables and their types.
\end{enumerate}
The modified top-level typed-AST, along with these extracted function definitions, are then used as the representation of the program until translation into the intermediate representation occurs.





\section{Optimisations on Closure Converted Typed AST}
I then perform two optimisations on this closure-converted typed-AST:
\begin{itemize}
	\item Direct Call Generation replaces applications of closures with direct calls of the corresponding function, if we can be sure the closure is always one of that function. This is the most effective optimisation I implement because it eliminates the overhead of closure calls, which involves wrapper functions to load the closure arguments from the closure before calling the actual function, and it can be applied to curried functions, enabling multiple arguments to be passed at once without constructing the intermediate closures.
	\item Tail Call Optimisation. A function is tail-recursive if it is recursive (calls itself), but only calls itself as the very last operation to return a result, and never performs additional computation after calling itself. Tail-recursive functions can be modified to a loop instead of calling themselves, which avoids the function call overhead of putting an additional frame on the stack.
\end{itemize}
Both of these are implemented through rule-based analysis of abstract values.

\subsection{Direct Call Generation}
Direct Call Generation walks the AST, diving into functions at the location of their make closure operation, determining which expressions and variables at different point are certainly closures for a specific function, and which variables are available at each point. When a closure application is encountered, we do the following:
\begin{enumerate}
	\item Check if the value being applied is known to be a closure to a specific function.
	\item Check if that function's closure variables are available. Because we do not rename closure variables inside functions, it suffices that if a variable e.g. \textinline{x#3} is available outside the closure, and is also a closure variable, then they refer to the same value.
	\item If we have a single application, we can replace directly with a call that provides the necessary closure variables.
	\item If we have a curried application, we must first determine the order to pass the arguments, and then we can pass all the arguments at once along with the closure variables.
\end{enumerate}

\subsection{Tail Call Optimisation}
Before we can apply tail call optimisation to a function, we must first check if it is tail-recursive. To do this we use a rule-based analysis with three possible `types': simple, tail-recursive and recursive. A simple function returns a value without calling itself, a tail-recursive function calls itself only as the last thing it does before it returns, and a recursive function calls itself and then performs additional computation afterwards.
\\\\
The rules are quite simple:
\begin{itemize}
	\item Constants and calls to other functions have type \textinline{text} providing all the arguments are \textinline{simple}
	\item Recursive calls to the same function have type \textinline{tailrec} providing all the arguments are \textinline{simple}.
	\item An expression that performs computation on its subexpressions (e.g. addition) has type \textinline{simple} if all it's arguments are \textinline{simple}, otherwise it is \textinline{recursive}.
	\item An expression that doesn't perform computation gives the worst type of it's subexpressions.
\end{itemize}
For instance, \camlinline{if simple then simple else tailrec} has type \textinline{tailrec} because it does not perform computation on the result of the \textinline{tailrec}, but \camlinline{if tailrec then simple else tailrec} has type \textinline{recursive} because it uses the result of the condition to determine which branch to choose.
\\\\
Once we have discovered a tail-recursive function, we then convert it to use a while loop as follows. Figure \ref{fig:tailrec} shows this in action:
\begin{enumerate}
	\item We create references for each argument, and for the result
	\item Inside the body of the while loop, we start by loading the arguments from the references
	\item We replace each tail call with loading the correct arguments into the references, and starting the loop body again (`continuing')
	\item If we manage to compute the result, we store it inside the result reference
	\item Once outside the loop, we simply dereference the result reference
\end{enumerate}
This gives us our iterative function body. It can also introduce `broken, unreachable' code, for instance the \camlinline{continue} statement is of type unit so it must return a unit type which is then assigned to result which is not of type unit. This code is clearly broken, but it is also unreachable as the \camlinline{continue} statement restarts the loop instead, so unreachable code analysis is required to eliminate it later. Equally broken is the creation of the result reference if our function does not return an integer, however this can be eliminated via dead-code elimination.

\begin{figure}[h]
\begin{minipage}{0.5\linewidth}
	\begin{minted}[linenos]{OCaml}
let rec fact n acc =
  if n = 0 then
    acc
  else
    fact (n - 1) (n * acc)
	\end{minted}
\end{minipage}
\begin{minipage}{0.5\linewidth}
\begin{minted}[linenos]{OCaml}
let fact n_in acc_in =
  let n_ref = ref n_in in
  let acc_ref = ref acc_in in
  let result_ref = ref 0 in
  while true do
    let n = !n_ref in
    let acc = !acc_ref in
    let result =
      if n = 0 then
        acc
      else
        (n_ref := n - 1;
        acc_ref := n * acc;
        continue)
    in
    result_ref := result
  done
  !result_ref
\end{minted}
\end{minipage}
	\caption{Before and after of tail-recursion optimisation}
	\label{fig:tailrec}
\end{figure}





\section{Intermediate Translation}
%\note{
%	What are the interesting cases?
%	\begin{itemize}
%		\item Mutual recursion: Create closures and then fill them up
%		\item Patterns: Generate code to both check them and destructure
%		\item Match Statements: Go through each block, exit the block early if match fails, until we finish a block and then can exit the whole match
%		\item Boxing of floats
%	\end{itemize}
%}

The original intermediate representation of the compiler was designed to target WebAssembly as simply as possible, and as such was structured and stack-based, with instructions pushing and popping values to the stack, and even containing lists of sub-instructions to evaluate arguments. While this approach allowed me to reach my success criteria quickly, it would have been very difficult to optimise for due to the need to keep track of what data is on the stack while doing analyses.
\\\\
Thus I introduced a new `variable-based' unstructured intermediate representation, where each instruction can take multiple variables as arguments and write a result to one variable, and structure is represented by special begin and end instructions for each structure.
\\\\
The full IR is quite large, but its instructions can be broken down into one of four categories:
\begin{itemize}
	\item Basic operations on variables, e.g. assigning a constant to a variable, copying a variable, and unary and binary operations
	\item Control instructions: those that mark the start and end of blocks, loops and if-else statements, and instructions for jumping out of these structures, as well as the special `fail' instruction.
	\item Memory operations, e.g. creating and loading from tuples/constructs/boxes and closures
	\item A closure-calling instruction, and a direct-call instruction.
\end{itemize}
The types used in the Intermediate Representation are shown in \ref{fig:ir}, with the full IR available in Appendix B.
\begin{figure}[h]
	\begin{minted}[linenos]{OCaml}
type itype =
(* Poly is the supertype of all types represented as i32 in WebAssembly *)
| It_poly | It_bool | It_int | It_pointer | It_unit
| It_float
| It_none (* No type, used for functions with no return type *)
	\end{minted}
	\caption{An overview of the types used in the Intermediate Representation}
	\label{fig:ir}
\end{figure}
Translation into the IR involves translating each function's expression into a list of these intermediate instructions. The top-level AST is translated into it's own sequence of instructions and packaged into a new special `init' function which must take no arguments and return no results, hence the inclusion of the \camlinline{It_none} type. Throughout the translation, we keep track of variables introduced (both temporary and named) and their types. Named variables in the top-level AST become global variables, while temporary variables introduced here become local variables of the init function.
\\\\
This transformation is handled in \textinline{intermediate.ml}, with two main functions:
\begin{itemize}
	\item \camlinline{transform_expr} Recursively transforms an expression into a list of instructions, and a variable for the result of the expression.
	\item \camlinline{transform_pat} This takes a pattern and a variable, and generates a list of instructions that will both test the structure of the value is stored in the variable against the pattern, and destructure this value into new variables. For instance in \camlinline{let (3, a) = x} we will generate code to test the first element of x's tuple to ensure that it is 3, and assign the second element to the variable a.
\end{itemize}
The majority of the transformation is relatively straightforward, however a few cases required additional thought:

\subsection{Closures and (Mutually) Recursive Functions}
Recursive functions require their own closure to call themselves recursively, and the problem is made worse by mutually recursive functions that all require access to each others' closures. Mutually recursive definitions can occur inside expressions, and hence it is not always the case that these closures would be available as global values. This left two possible options:
\begin{enumerate}
	\item Create a new closure for the recursive function or its mutually recursive friends whenever a recursive call occurs, by using the closure variables passed in as arguments to the function.
	\item Include recursive closures inside their own `closure variables'.
\end{enumerate}
I chose the second option because the first option would have been both less efficient, and more difficult to implement. In order to include closures inside their own `closure variables', there are two instructions to create closures. The first one, \textinline{Iins_newclosure}, creates a `new' empty closure for the specified function, while \textinline{Iins_fillclosure} takes an existing empty closure, and fills it up using the variables provided, one of which can be the variable storing the currently empty closure or currently empty closures of mutually recursive friends.

\subsection{Match Statements}
Match statements are handled through nested blocks. The outer block represents the entire match statement, while inner blocks represent individual cases of the match statement. In addition, a temporary variable is introduced to store the result of the entire match statement. The code for an individual case is as follows
\begin{enumerate}
	\item Code for the case's pattern, modified to replace fail instructions (indicating the pattern was not matched), with instructions to jump out of the case block and thus proceed with the next case.
	\item Code for the case's expression
	\item An instruction to copy the expression's result into the match statement's result
	\item An instruction to jump out of the outer match statement block, and thus skip over the remaining cases
\end{enumerate}
In addition, a fail instruction is included after all the cases inside the match block to ensure that if no cases match, execution terminates with a failure.

\subsection{Polymorphism}
Polymorphism is implemented through the special intermediate type \camlinline{It_poly}, which indicates that a variable holds a polymorphic value. This type is used for the arguments and results of all closures, as when an arbitrary closure is called we cannot be sure whether it is a polymorphic function or not and thus must assume it is. It is also used for the contents of tuples, which can also be used polymorphically. 
\\\\
Types that are represented as 32-bit integers in WebAssembly, such as integers, units, booleans and pointers, are all subtypes of this polymorphic type, and hence values of these types can be directly stored in these variables and passed as arguments to functions. Floating point numbers however are not a subtype, as the assignment of a floating point value to an integer variable in WebAssembly is not allowed. Hence we have to `box' these floating-point values, meaning that we allocate a space in memory and store the floating-point value there, passing the pointer to this location instead.
\\\\
The result of the intermediate translation is a list of function objects, which contain the name, intermediate code, and list of variables used by that function. In addition, a list of global variables is produced.





\section{Optimisations on the IR}
%\note{
%	\begin{itemize}
%		\item Unreachable code elimination (needed due to tail-call optimiser's generation of broken unreachable code e.g. assign unit to float because the `break' statement / tail-call gives a unit type)
%		\item Copy propagation (if we have y=x, followed by z = y+y we can replace y with x. By far the most complex of the IR optimisations because we need to know both that the most recent definition of y is y=x, and that x has not changed since then)
%		\item Tuple load elimination (If we know a tuple is (x,y,z), then instead of loading it when matching with (a,b,c), we can just do a=x, b=y, z=c directly. Also works on constructs)
%		\item Dead code elimination (removes useless units, and also tuple creation when tuple usage eliminated by tuple load elimination)
%		\item Ref elimination (eliminates refs that are used as mutable variables only, by using mutable variables)
%	\end{itemize}
%}

Before code generation, a second round of optimisations is performed. These optimisations are predominantly data-flow analyses. Data-flow analysis concerns the movement of data through program code, asking questions like `Will the value assigned here be used?' (Live Variable Analysis) and `Where could the current value of this variable have been assigned?' (Reaching Definitions). Unlike the first round of optimisations, which improved execution time at the expense of code size, these optimisations both improve execution time and decrease code size.
\\\\
Before data-flow analysis can be performed, function code must first be deconstructed into a basic-block graph. A basic block is a sequence of instructions that will always begin execution at the first instruction, and finish at the last, with no jumps out in-between. A conditional jump at the end of one basic block might mean that there are several possible blocks that could be executed next, and hence a graph is required to keep track of these.
\\\\
My approach to construct this graph is to first build a jump-table which records which lines have jumps and where they might jump to. From this we can extract all locations with incoming or outgoing jumps, and use these to split the code into these blocks. We can then use the table to determine the outgoing edges of the basic blocks, and from those record in each basic block its possible predecessor or successor basic blocks.

\subsection{Data-Flow Analysis}
My compiler then implements three forms of data-flow analysis for use in optimisations:
\begin{itemize}
	\item Live Variable Analysis: For each instruction $i$, the variable $x$ is syntactically live at that instruction if there exists an instruction $j$ that is reachable from $i$ without passing any assignments to $x$ and uses the value of $x$. There is another definition of liveness, semantic liveness, whereby a variable is life if it's value will affect the input/output behaviour of the program at some point in the future. As this is generally uncomputable, syntactic liveness is used as a safe approximation.
	\item Reaching Definitions: For each instruction $i$, a set for each variable $x$ of instructions $j$ that assign to $x$ and for which there is a path from $j$ to $i$ without passing another assignment of $x$.
	\item `Available Assignments': For each instruction $i$, a set for each variable $x$ of variables $y$ that, if assigned in an instruction that uses the value of $x$, the value of $x$ has not changed since that assignment to $y$ (but $y$ itself may have been reassigned). This analysis is similar to the more conventional Available Expressions analysis, except that instead of storing expressions for which their operands have not changed since they were computed, it stores the variables assigned from these expressions.
	\note{Someone must have done this before? Otherwise is there a better name for this? Also while yes it is similar to Available Expressions, it does use weird positive/negative sets and big-union instead of a single-set and big-intersection. \begin{itemize}
			\item Add y to x's sets: add to positive, remove from negative
			\item x is assigned, so all variables that might use it are now invalid: move everything from positive set into negative set
			\item Union: union the negative sets, new positive set is union of positive sets minus union of negative sets
			\item No proof that it works? Will the examiners hate me for this? The argument below convinces me that it works, but is not a proof.\\\\What are the cases: 1) when we have multiple basic blocks leading to the current one, 2) Assignment to x reachable from y = x. 3) Assignment to x not reachable from y = x (e.g. example below) 4) Multiple y=x where one also has an assignment to x afterwards
			\\\\
			What would a case look like where it didn't work? x's positive set includes y, but the value of x could have changed after the assignment y = x. The assignment to x clearly has to come after the assignment y = x, and thus y = x can clearly reach this assignment, so either y is in x's positive set and it gets negated, or y is already in x's negative set. There is no way that y can't be in either set, which is the only case that would cause an issue.
	\end{itemize}}
\end{itemize}
An example of these analyses is shown in figure \ref{fig:dataflow}. Line 0 is used to refer to the beginning of the function. Note that on line 8, $y$ is still in the available assignment set for $x$. This is correct, as in all cases where $x$ is assigned to $y$ (as in line 3), $x$ is not modified after this assignment.
\begin{figure}[h]
	\begin{minipage}{0.3\linewidth}
	\begin{minted}[linenos, firstnumber=0]{python}
myfunc(x):
  y = 9
  if cond:
    y = x
    y = 10
  else:
    x = 2
    y = 4
  return y
	\end{minted}
	\end{minipage}
	\begin{minipage}{0.7\linewidth}
		\begin{center}
			\begin{tabular}{|c|c|c|c|}
				\hline
				Line & LVA (at end) & RDs (at start) & AAs (at start) \\
				\hline
				1 & $x$ & $(x, \{0\})$ & \\
				\hline
				3 & & $(x, \{0\}), (y, \{1\})$ & \\
				4 & $y$ & $(x, \{0\}), (y, \{3\})$ & $(x, \{y\})$ \\
				\hline
				6 & & $(x, \{0\}), (y, \{1\})$ & \\
				7 & $y$ & $(x, \{6\}), (y, \{1\})$ & \\
				\hline
				8 & & $(x, \{0, 6\}), (y, \{4, 7\})$ & $(x, \{y\})$ \\
				\hline
			\end{tabular}
		\end{center}
	\end{minipage}
	\caption{An example of the three data-flow analyses}
	\label{fig:dataflow}
\end{figure}
\\\\
For each of these analyses, it is possible to define data-flow equations where we compute these sets for a line by taking either the union or intersection of the sets from all lines that are direct predecessors or successors of the line. The equations for reaching definitions analysis are as follows:
\begin{align*}
\text{in-defs}(i) \quad=&\quad \bigcup_{p\ \in\ \text{pred}(i)} \text{out-defs}(p) \\
\text{out-defs}(i) \quad=&\quad (\text{in-defs}(i) \setminus \text{undef}(i)) \cup \text{def}(i)
\end{align*}
where $\text{undef}(i)$ and $\text{def}(i)$ depend on the instruction $i$. If $i$ does not assign a variable, they are both empty, otherwise if $i$ assigns to $x$, then $\text{undef}(i)$ is all previous assignments to $x$, while $\text{def}(i)$ is the new assignment to $x$, or $(x, \{i\})$. Similar equations can be given for the other analyses, using different def and undef functions: available assignments uses two sets a `positive set' and `negative set' for each variable (\note{Explain this more}), and live-variables operates backwards using a union of the successor lines to compute the out-lva, and then computing in-lva from that.
\\\\
The algorithm for implementing these data-flow analysis follows from these equations: Put the set for each line to empty initially, and iterate over all basic blocks applying these equations (all of our analyses use big-union, so there is no issue with intersection) until there is an iteration in which nothing changes, and then return the result. \note{Explain this better as well. Perhaps give pseudocode or steps}

\subsection{Transformations From These Analyses}
These analyses allow me to implement a number of transformations:
\begin{itemize}
	\item Unreachable Code Elimination. Unreachable code is code that will never be executed, and hence it is safe to remove. A safe approximation of unreachable code is given by basic blocks that have no predecessors (and are not the first block of the function), and I eliminate the code in these blocks, taking care to retain structural instructions such as those that signify the end of an if-statement or `block'.
	\item Dead Code Elimination. Dead code is code that may be executed, but produces a result that will never be used. Live Variable Analysis can be used, as if a variable is not live at the end of a line where it is assigned, that assignment is never used, and thus if that line contains no side-effects it can be eliminated.
	\item Copy Propagation. Using the results of Reaching Definitions, it becomes possible to check when we see a use of a variable $y$, if the most recent definition of $y$ is $y = x$, and using Available Assignments, we can check that $x$'s value hasn't changed since that assignment. In this case, it is safe to replace a usage of $y$ with $x$. If all usages of $y$ are eliminated, the assignment $y = x$ can then be eliminated with Dead Code Elimination.
	\item Tuple-load Elimination. Similar to copy-propagation, if we encounter the loading of a variable from a tuple $t$, e.g. loading of the 2nd element of a 3 element tuple, we can use Reaching Definitions to see if there is a unique assignment of $t$. If there is and it is of the form $t = (x_1, ..., x_n)$, we can then check using Available Assignments to make sure the $x_i$ we want has not been modified since, and if so we replace the load with $x_i$. If this eliminates all usages of $t$, Dead Code Elimination can then eliminate the construction of the tuple, which can result in a significant speedup. This optimisation is particularly useful when dealing with match statements over multiple variables, as it eliminates the creation of the tuple in that case.
\end{itemize}

\subsection{Other Transformations}
In addition, I perform a transformation I call as `ref elimination'. If a reference is created inside a function, and then only used when it is updated or dereferenced, without being passed as an argument to another function, stored in memory or returned from the function, then the reference can be eliminated and instead a mutable variable is used, which eliminates the memory operations that go with using the reference, and results in a small but noticeable speedup.

\subsection{Ordering of Transformations}
Individually, each transformation is looped until it does not result in a change to the code. However, the order in which the transformations are performed can have a great effect on their effectiveness, for instance if Dead Code Analysis were run only before Tuple-load Elimination, we would not be able to eliminate the unused tuple assignments resulting from Tuple-Load Elimination.
\\\\
Therefore, I order the transformations by running each in order, but then running transformations that might benefit from an earlier transformation again after the earlier transformation. This results in some transformations being run many times, however I do not consider that a problem as the compiler is fast to execute for all samples.

\section{Code Generation}
% \note{The stack code generator removes redundant variable saving/loading. Also representation of closures and tuples/constructs/refs}

The code generator is responsible for generating a WebAssembly Text Format (.wast) file of the program based on its global variables and function definitions in the IR. A skeleton file is used which contains a runtime common to all programs, which is then filled in with the specific details of this program. A stack-based algorithm is used to transform the variable-based IR into stack-based WebAssembly code, which results in much smaller code than the original simple algorithm I used. For each function, we also include a wrapper function that a closure call invokes before the full function is called.

\subsection{WebAssembly Text Format}
A WebAssembly Text Format file contains the definition of a WebAssembly module as an S-expression. S-expressions are a representation of nested lists as either an atom, such as \wainline{global} or \wainline{i32}, or a list of S-expressions, such as \wainline{(big (sub sexpr) sexpr)}, which is a list containing the atom `big', the list \wainline{(sub sexpr)} and the atom `sexpr'. A module is a list with the atom \wainline{module} followed by definitions, of which there are several useful types:
\begin{itemize}
	\item Global variable, e.g. \wainline{(global $name (export "export_name") (mut i32) (i32.const 0))}, with a name, a type, and code to initialize the variable, which for my compiler is always a single constant instruction.
	\item Function, e.g.\wainline{(func $name (export "export_name") (param $closure i32) (param $arg i32) (result i32) (local $myloc i32) code_1 ... code_n )}, where \wainline{code_1} to \wainline{code_n} are S-expressions representing instructions (but each instruction will be multiple S-expressions in a row). Functions can have any number of parameters or local variables, and the result S-expression can be omitted if the function has no result.
	\item Memory, e.g. \wainline{(memory (export "memory") 1)} gives access to an expandable memory area. Currently WebAssembly modules can only have one memory, but in the future they may support multiple.
	\item Table, e.g. \wainline{(table (export "wrapper_functions") funcref (elem $funcname_1 ... $funcname_n))}. Currently the only use of tables is to define the indirect function call table, an ordered list of function names. Indirect call instructions take as an argument the index of a function in this table.
\end{itemize}
Instructions themselves are a list of atoms, which can be included within a function's S-expression. To give a feel for the instruction-format, here are a few commonly used instructions:
\begin{itemize}
\item Variable get/set: \wainline{local/global.get/set/tee $varname} for loading and saving variables to/from the stack. \wainline{local.tee} is a special instruction that saves a local variable without popping it from the stack, to which there is no \wainline{global.tee} counterpart.
\item Unary/binary operations, e.g. \wainline{i32.add} These always specify a type and an operation, and will take either one or two arguments off the stack, and push the result to the stack. Code is verified by the WebAssembly execution environment when it is loaded to ensure that the stack will always contain enough values of the required type.
\item Control instructions, e.g. \wainline{block $block_name inner_code... end $block_name}. The inner code cannot access variables on the stack outside the block. Blocks can be executed early by conditional or unconditional branch instructions, e.g. \wainline{br_if $block_name} to leave the block with name `block\_name'. Blocks have a result type, and for my compiler that type is always none, and hence I must ensure that whenever we leave the block (either by jump or at the end), there is nothing on the stack. \wainline{loop}s are similar to blocks, but a branch to the loop's name will restart the loop block.
\end{itemize}

\subsection{Runtime System}
My compiler inserts a small runtime at the start of the WebAssembly module, which contains a function \wainline{$malloc} to allocate a memory region of a fixed size and return the address of its start, and two global variables \wainline{$mem_idx} and \wainline{$mem_max} which keep track of the next unallocated index in memory and the current size of memory respectively. If the memory becomes full, \wainline{$malloc} will call a special WebAssembly function to grow the memory.
\\\\
In addition, I include a function \wainline{$call_closure}, which is not used by the WebAssembly code, but allows calling of a WebAssembly closure from the JavaScript environment.

\subsection{Code Generation of Functions}
For each function in the IR, three things are added to the WebAssembly module:
\begin{itemize}
	\item A wrapper function, which is invoked by closure calls to that function. The type of this function is always \wainline{(param i32) (param i32) (result i32)}, where the first argument is the closure and the second is the (potentially boxed) argument. We load the closure variables onto the stack, unbox the argument if necessary, and then invoke the actual function, boxing the result of this function if required.
	\\\\
	These wrapper functions are needed for every function because we need to know the exact type of the function when we invoke it indirectly, and functions can have variable numbers of closure variable arguments, so it is not possible to write a single function that can invoke any closure without these wrapper functions.
	\\\\
	The alternative would be to inline a function's code inside the wrapper function, which would be undesirable is it would prohibit direct calls.
	\item The wrapper function is added to the funcref table, which allows indirect calls to it.
	\item The actual function itself. This can have a varying number of arguments of either integer or float type, and may return either an integer or a float. I use a stack-based algorithm to optimise the code of the function, significantly reducing its size.
\end{itemize}

\subsection{The Stack Code Generator}
%\note{Fun cases:
%\begin{itemize}
%	\item Closure call
%	\item Memory access
%	\item Mostly straightforward for the middle bit
%\end{itemize}}
%
%\note{
%	\begin{itemize}
%	\item Explain three parts, and middle and end parts
%    \item Explain gen blocks and the gen block stack
%    \item Explain looking up LVOs backwards, which can pop things from the stack
%    \item Explain how we find a stack variable, and no-bypass, and tee-ing
%    \end{itemize}
%}

The stack-based code generation algorithm operates on each basic block of the function's IR. We iterate through the code inside the basic block, working with a stack of \camlinline{gen_block}, or `generated blocks' (GB) to produce the final WebAssembly code. These GB are used to determine how to best load the variables that an instruction needs by making use of the stack and thus minimising use of \wainline{local.get} and \wainline{local.set} instructions.
\\\\
The definition of a \camlinline{gen_block} is shown in figure \ref{fig:genblock}. The \camlinline{code} is the WebAssembly code of the block, \camlinline{out_stack} is the optional variable who's value is left on top of the stack after the GB is executed, and \camlinline{assigned} is the set of all variables that are assigned during the block, including the variable that is output onto the stack if it exists.
\begin{figure}[h]
	\begin{minted}[linenos]{Ocaml}
type gen_block = {
  start_line: int; (* Start line in the IR code *)
  code: string; (* WebAssembly code *)
  out_stack: ivariable option; (* Output to stack *)
  assigned: ivariable Set.t; (* Variables that are assigned *)
  teed: bool; (* Is the stack variable teed *)
}
	\end{minted}
	\caption{The definition of a gen\_block}
	\label{fig:genblock}
\end{figure}
\\\\
The code generation for each instruction is split into three parts:
\begin{enumerate}
	\item \textbf{The instruction produces a `load variable order'}, where it specifies which variables it needs, and for each variable whether it wants it on the stack or just to be available (meaning \wainline{local.get} for that variable will produce the most up to date value, rather than that value being on the stack)
	
	\item \textbf{Code to execute the instruction}, e.g. perform the arithmetic operation. The function to generate this code is provided with a list of code blocks corresponding to the `load variable order', which is interspersed with the instructions code.
	\\\\
	As an example, consider generating code for the instruction to create a new tuple. Each tuple argument must be saved to memory, but the WebAssembly memory store instruction \wainline{i32.store offset=n} requires the data to be on top of the stack, with the base address below that. As WebAssembly has no `swap' instruction to swap the top two items on the stack, we must ensure that we put the base address on the stack before the data, thus when the code is generated, we insert the code corresponding to the variable we want from the lvo in between code to put the base address on the stack, and the store instruction.
	
	\item \textbf{Each instruction can have one of three results}: either it does no assignment (e.g. a branch instruction), it `assigns' to a variable and the result is put on top of the stack (meaning the value stored in the actual variable is the old value), or it assigns to a variable and saves it directly in that variable.
\end{enumerate}
After we have generated code for an instruction, the resulting code is packaged into a GB. The tricky part of the algorithm then becomes satisfying instructions' `load variable orders':
\begin{itemize}
	\item \textbf{When a variable is wanted on the stack}, and we have a GB that puts that variable on the stack, we go up the GB stack, changing those GB in-between so that they no longer put their variables on the stack but save them instead. Thus, the variable we want is now on the top of the stack, and that GB and all following blocks are popped from the GB stack and included in the code to load the variable.
	\\\\
	Load variable orders are processed backwards, so the last variable the instruction wants is handled first. Because we want the most up-to-date version of all the variables, when we unwind the GB stack we prohibit going past assignments to variables that are earlier in the load variable order. In that case, the GB that writes the variable to the stack is modified to store the variable instead, and we use \wainline{local.get} to load the variable to the stack.
	\item \textbf{If we want a stored variable}, and we have a GB that puts that variable on the stack, we change it to save that variable.
	\item \textbf{If the variable is already saved}, then we can simply load it with a \wainline{local.get} instruction
\end{itemize}
This algorithm has several additional edge cases, for instance some control instructions require the stack to be emptied.
\\\\
At the end of this, we modify all resulting GBs to save their variables instead of writing to the stack (we assume the values will be needed in another basic block), and concatenate their code together to produce the final code of the function.




\section{Summary}
To summarise the components of the compiler:
\begin{itemize}
	\item Lexing and parsing is done by the OCaml compiler library, producing an untyped OCaml AST
	\item The type-checker uses Hindley-Milner type inference to produce a typed-AST from the untyped AST
	\item In lambda lifting and closure conversion, we extract function definitions, replacing their original definition in the AST with a `make closure' function
	\item We then look through these extracted functions to replace closure calls with direct calls wherever possible
	\item Tail-recursive functions are detected and converted to be iterative using while-loops
	\item These extracted function definitions and top-level AST are then converted to the Intermediate Representation, which produces a list of intermediate functions including a special init function.
	\item Data-flow optimisations are then applied within these functions, performing Dead and Unreachable Code Elimination, Copy Propagation, `Tuple Load Elimination' and `Ref Elimination'
	\item WebAssembly code is generated, deploying a stack-based algorithm to produce significantly shorter WebAssembly Text Format files with fewer local get/set operations.
\end{itemize}


\section{Repository Overview}
\begin{minipage}{\linewidth}
\dirtree{%
	.1 root/.
	.2 documents/.
	.2 compiler/.
	.3 base/lib/.
	.3 types/lib/.
	.3 transform/lib/.
	.3 codegen/lib/.
	.3 testing/.
	.3 samples/.
	.4 benchmark/.
}
\end{minipage}
\\\\
The \textinline{documents} directory contains documents such as this dissertation, the proposal, progress report and presentation, while the code is inside the \textinline{compiler} directory:
\begin{itemize}
	\item The \textinline{compiler} directory itself contains the toplevel \textinline{dune} and \textinline{dune-project} build files, and a file called \textinline{toplevel.ml} which is the top-level compiler program that passes the command line arguments and invokes the submodules.
	\item \textinline{base/lib} contains code common to all other modules
	\item \textinline{types/lib} contains the type-checker and definition of the typed-ast
	\item \textinline{transform/lib} contains closure conversion, transformation to the IR and optimisations
	\item \textinline{codegen/lib} contains the code generator
	\item \textinline{testing} contains the end-to-end tester and benchmark system, in a file \textinline{end2end.js}, and supporting JavaScript files
	\item \textinline{samples} contains OCaml samples and corresponding JSON files detailing tests to be performed on them
	\item \textinline{samples/benchmark} contains samples used as benchmarks and JSON files detailing the benchmarks
\end{itemize}



% CHAPTER
\clearpage
\chapter{Evaluation}
\note{This is where Assessors will be looking for signs of success and for evidence of thorough and systematic evaluation as discussed in Section 8.3. Sample output, tables of timings and photographs of workstation screens, oscilloscope traces or circuit boards may be included. A graph that does not indicate confidence intervals will generally leave a professional scientist with a negative impression.}

\note{As with code, voluminous examples of sample output are usually best left to appendices or omitted altogether.}

\note{There are some obvious questions which this chapter will address. How many of the original goals were achieved? Were they proved to have been achieved? Did the program, hardware, or theory really work?}

\note{Assessors are well aware that large programs will very likely include some residual bugs. It should always be possible to demonstrate that a program works in simple cases and it is instructive to demonstrate how close it is to working in a really ambitious case.}

\section{Success Criteria}
\note{Say that it has been satisfied, and list the success criteria}

\section{End To End Tester}
\note{Could do with a -all mode to test all combinations of disabled / enabled optimisations}

\section{Benchmarks / Benchmark Driven Optimisations}
\note{
	\begin{itemize}
		\item Gcd: main target of optimisations. Went from 122ms to 8ms thanks to all of them.
		\item others: they improve (e.g. especially with direct call generation), but not as much with the other optimisations
		\item Make some performance graphs of the final result vs other execution engines, and of different optimisations enabled/disabled (can do both time and memory for that).
	\end{itemize}
}

% TODO DONT FORGET ALL THE WORK IN MAKING THE CODE SHORTER

% TODO HOW IT WAS EVALUATED
% TODO TEST SYSTEM AND BENCHMARKS




\clearpage
\chapter{Conclusion}
\note{This chapter is likely to be very short and it may well refer back to the Introduction. It might properly explain how you would have planned the project if starting again with the benefit of hindsight. }

\note{
	\begin{itemize}
		\item SSA based IR?
		\item More optimisations e.g. reverse copy-propagation, mutually recursive tail call optimisation, better match statements, better code generator that can re-order non side effecting instructions, speedy append implementation
		\item More feature support: strings, records and mutable records, modules, named/optional function arguments, using the operators as function arguments (e.g. List.reduce ~f:(+) nums to sum a list)
	\end{itemize}
}

% TODO CONCLUDE THE DOCUMENT
% TODO IVE SUCCESSFULLY MET SUCCESS CRITERIA, SUMMARY OF EVERYTHING
% TODO WHAT LESSONS DID I LEARN
% TODO WHAT WORK COULD YOU DO IN THE FUTURE




\clearpage

%%%%%%%%%%%%%%%%%%%%%%%%%%%%%%%%%%%%%%%%%%%%%%%%%%%%%%%%%%%%%%%%%%%%%
% the bibliography

\addcontentsline{toc}{chapter}{Bibliography}
\printbibliography[title={Bibliography}]
\clearpage

%%%%%%%%%%%%%%%%%%%%%%%%%%%%%%%%%%%%%%%%%%%%%%%%%%%%%%%%%%%%%%%%%%%%%
% the appendices
\appendix
% Assessors like to see some sample code or example circuit diagrams, and appendices are the sensible places to include such items. Accordingly, software and hardware projects should incorporate appropriate appendices. Note that the 12,000 word limit does not include material in the appendices, but only in extremely unusual circumstances may appendices exceed 10-15 pages - if you feel that such unusual circumstances might apply to you you should ask your Director of Studies and Supervisor to apply to the Chairman of Examiners. It is quite in order to have no appendices. Appendices should appear between the bibliography and the project proposal. 

\chapter{The First Appendix}

Things in appendix A / the Typed AST?


\clearpage

\chapter{The Intermediate Representation}
\begin{minted}{OCaml}
type itype =
| It_poly | It_bool | It_int | It_pointer | It_unit
| It_float
| It_none

type iftype = itype * itype (* Function type *)
type ituptype = itype list (* Tuple type *)

type iunop =
| Iun_neg (* Negate *)
| Iun_eqz (* Equals zero *)

type ibinop =
| Ibin_add | Ibin_sub | Ibin_mul | Ibin_div | Ibin_rem (* Arithmetic *)
| Ibin_and | Ibin_or (* Logic *)
| Ibin_eq | Ibin_ne (* Equality *)
| Ibin_lt | Ibin_le | Ibin_gt | Ibin_ge (* Comparison *)

type iscope = Isco_local | Isco_global (* Variable scope *)

type ivariable = iscope * string

type iinstruction =
(* Create a new var from a constant *)
(* type of var, name of var *)
| Iins_setvar of itype * ivariable * string
(* Copy a var into another *)
(* type of var, name of new var, name of old var *)
| Iins_copyvar of itype * ivariable * ivariable
(* Return var *)
(* type of var, name of var *)
| Iins_return of itype * ivariable
(* Unary operation using one argument value *)
(* type of operand, unary operation, result var, input var *)
| Iins_unop of itype * iunop * ivariable * ivariable
(* Binary operation using two argument values *)
(* type of operands, binary operation *)
| Iins_binop of itype * ibinop * ivariable * ivariable * ivariable
(* Make a new closure for specified function and tuple type *)
(* type of function, name of function, type of closure vars, result var *)
| Iins_newclosure of iftype * string * ituptype * ivariable
(* Fill a closure with a list of vars *)
(* type of closure vars, name of var, list of vars to copy in *)
| Iins_fillclosure of ituptype * ivariable * ivariable list
(* Call closure in var, passing in an argument *)
(* type of function, output var, closure var, var for argument *)
| Iins_callclosure of iftype * ivariable * ivariable * ivariable
(* Directly call a function, passing multiple arguments *)
(* output var, name of function, type of args, arg vars *)
| Iins_calldirect of ivariable * string * ituptype * (ivariable list)
(* Start a block *)
(* name of block *)
| Iins_startblock of string
(* End a block *)
(* name of block *)
| Iins_endblock of string
(* Exit from the named block *)
(* name of block *)
| Iins_exitblock of string
(* Exit from the named block if var is true *)
(* name of block *)
| Iins_exitblockif of string * ivariable
(* Start an if statement *)
(* name of block, condition var *)
| Iins_startif of string * ivariable
(* Else clause of an if statement *)
(* name of block *)
| Iins_else of string
(* End an if statement *)
(* name of block *)
| Iins_endif of string
(* Starts a loop, loops until an exitblock or exitblockif *)
(* Name of escape block (to break to), name of loop block (to continue to) *)
| Iins_startloop of string * string
(* Ends a loop *)
(* Name of break block, name of continue block *)
| Iins_endloop of string * string
(* Create a tuple of the given vars *)
(* type of tuple, result var, argument vars *)
| Iins_newtuple of ituptype * ivariable * ivariable list
(* Load tuple's value at index i *)
(* type of tuple, index in tuple, output var, tuple var *)
| Iins_loadtupleindex of ituptype * int * ivariable * ivariable
(* Create a construct of the given id and vars *)
(* type of construct arguments, result var, id of construct, argument vars *)
| Iins_newconstruct of ituptype * ivariable * int * ivariable list
(* Load construct's value at index i *)
(* type of construct arguments, index in arguments, output var, construct var *)
| Iins_loadconstructindex of ituptype * int * ivariable * ivariable
(* Load construct's ID *)
(* output var, construct var *)
| Iins_loadconstructid of ivariable * ivariable
(* Create a mutable memory box for a value *)
(* unboxed type, result var, var to box *)
| Iins_newbox of itype * ivariable * ivariable
(* Update a mutable memory box with a new value *)
(* unboxed type, boxed var, unboxed var *)
| Iins_updatebox of itype * ivariable * ivariable
(* Load a value from a mutable memory box *)
(* unboxed type, target var, boxed var *)
| Iins_unbox of itype * ivariable * ivariable
(* Fail, suspending execution *)
(* No parameters *)
| Iins_fail
\end{minted}

\clearpage
\chapter{Walkthrough Compilation of GCD}
\note{Tobias' proposal that I have an example program and show it at each step of the compilation. Could be a good idea but the code is massive, so this would easily extend to about 10+ pages. Also might require renaming some of the temporaries to make the code easier to read}
\\\\
\note{Would go something like this:
\begin{itemize}
	\item Type-checker (the missing types would be quite obvious so this could be skipped?)
	\item Closure conversion: gcd and gcd-app are now extracted functions
	\item Direct Call generation: gcd-app now directly calls itself instead of applying gcd and then applying the resulting closure
	\item Tail Recursion Optimisation: gcd-app is now a while loop
	\item Intermediate Representation: here is what gcd-app looks like in the IR
	\item Unreachable Code: Look the broken code is gone
	\item Copy Propagation: Fewer copies!
	\item Tuple-Load: The match statement doesn't load from the tuple
	\item Dead-Code: Now the tuple is gone so we use less memory
	\item Ref Elimination: No more memory usage
	\item Code-gen: Here's the WASM code
\end{itemize}
}

GCD Function:
\begin{minted}{OCaml}
let rec gcd a b =
  match (a, b) with
  | (0, y) -> y
  | (x, 0) -> x
  | _ ->
      if a > b then
        gcd (a - b) b
      else
        gcd (b - a) a
\end{minted}

After IR Optimisations:
\begin{minted}{text}
Function $$f_gcd-app:
-type: int -> int
-vars:
$temp_25 -> $temp_25 (int)
$temp_21 -> $temp_21 (int)
$temp_17 -> $temp_17 (int)
$temp_16 -> $temp_16 (bool)
$temp_15 -> $temp_15 (bool)
$temp_14 -> $temp_14 (int)
$temp_10 -> $temp_10 (bool)
$temp_9 -> $temp_9 (int)
$temp_7 -> $temp_7 (int)
$temp_4 -> $temp_4 (poly)
$temp_1 -> $temp_1 (poly)
$temp_0 -> $temp_0 (poly)
$arg -> $arg (int)
arg_gcd_in -> $arg_gcd_in (int)
-code:
copyvar poly local.$temp_0 local.$arg_gcd_in
copyvar poly local.$temp_1 local.$arg
startloop $block_0 $block_1
copyvar poly local.$temp_4 local.$temp_0
startblock $block_2
startblock $block_3
setvar int local.$temp_9 0
binop int ne local.$temp_10 local.$temp_0 local.$temp_9
exitblockif $block_3 local.$temp_10
copyvar int local.$temp_7 local.$temp_1
exitblock $block_2
endblock $block_3
startblock $block_4
setvar int local.$temp_14 0
binop int ne local.$temp_15 local.$temp_1 local.$temp_14
exitblockif $block_4 local.$temp_15
copyvar int local.$temp_7 local.$temp_0
exitblock $block_2
endblock $block_4
startblock $block_5
binop int gt local.$temp_16 local.$temp_0 local.$temp_1
startif $block_6 local.$temp_16
binop int sub local.$temp_17 local.$temp_0 local.$temp_1
copyvar poly local.$temp_0 local.$temp_17
exitblock $block_1
else $block_6
binop int sub local.$temp_21 local.$temp_1 local.$temp_0
copyvar poly local.$temp_0 local.$temp_21
copyvar poly local.$temp_1 local.$temp_4
exitblock $block_1
endif $block_6
copyvar int local.$temp_7 local.$temp_25
exitblock $block_2
endblock $block_5
fail
endblock $block_2
exitblock $block_0
endloop $block_0 $block_1
return int local.$temp_7
\end{minted}

WebAssembly
\begin{minted}{LISP}
(func $$f_gcd-app (export "gcd")
	(param $arg_gcd_in i32)
	(param $arg i32)
	(result i32)
	(local $temp_0 i32)
	(local $temp_1 i32)
	(local $temp_4 i32)
	(local $temp_7 i32)
	(local $temp_25 i32)
	local.get $arg_gcd_in
	local.set $temp_0
	local.get $arg
	local.set $temp_1
	block $block_0
	loop $block_1
	local.get $temp_0
	local.set $temp_4
	block $block_2
	block $block_3
	local.get $temp_0
	i32.const 0
	i32.ne
	br_if $block_3
	local.get $temp_1
	local.set $temp_7
	br $block_2
	end $block_3
	block $block_4
	local.get $temp_1
	i32.const 0
	i32.ne
	br_if $block_4
	local.get $temp_0
	local.set $temp_7
	br $block_2
	end $block_4
	block $block_5
	local.get $temp_0
	local.get $temp_1
	i32.gt_s
	if $block_6
	local.get $temp_0
	local.get $temp_1
	i32.sub
	local.set $temp_0
	br $block_1
	else $block_6
	local.get $temp_1
	local.get $temp_0
	i32.sub
	local.set $temp_0
	local.get $temp_4
	local.set $temp_1
	br $block_1
	end $block_6
	local.get $temp_25
	local.set $temp_7
	br $block_2
	end $block_5
	unreachable
	end $block_2
	br $block_0
	end $block_1
	end $block_0
	local.get $temp_7
)
\end{minted}

\clearpage

\chapter{Project Proposal}
\clearpage

\input{propbody}

\end{document}
