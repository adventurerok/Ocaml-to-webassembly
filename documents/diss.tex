% The master copy of this demo dissertation is held on my filespace
% on the cl file serve (/homes/mr/teaching/demodissert/)

% Last updated by MR on 2 August 2001

\documentclass[12pt,twoside,notitlepage]{report}

\usepackage{a4}
\usepackage{verbatim}

\usepackage{minted}
%\usemintedstyle{colorful}
\setmintedinline{breaklines}

\newcommand{\textinline}{\mintinline{text}}
\newcommand{\cinline}{\mintinline{C}}
\newcommand{\camlinline}{\mintinline{OCaml}}

\input{epsf}                            % to allow postscript inclusions
% On thor and CUS read top of file:
%     /opt/TeX/lib/texmf/tex/dvips/epsf.sty
% On CL machines read:
%     /usr/lib/tex/macros/dvips/epsf.tex



\raggedbottom                           % try to avoid widows and orphans
\sloppy
\clubpenalty1000%
\widowpenalty1000%

\addtolength{\oddsidemargin}{6mm}       % adjust margins
\addtolength{\evensidemargin}{-8mm}

\renewcommand{\baselinestretch}{1.1}    % adjust line spacing to make
                                        % more readable

\usepackage[backend=bibtex, style=alphabetic, sorting=ynt]{biblatex}
\addbibresource{refs.bib}

\begin{document}



%%%%%%%%%%%%%%%%%%%%%%%%%%%%%%%%%%%%%%%%%%%%%%%%%%%%%%%%%%%%%%%%%%%%%%%%
% Title


\pagestyle{empty}

\hfill{\LARGE \bf Paul Durbaba}

\vspace*{60mm}
\begin{center}
\Huge
{\bf Compiling OCaml to WebAssembly} \\
\vspace*{5mm}
Diploma in Computer Science \\
\vspace*{5mm}
Robinson College \\
\vspace*{5mm}
May 2020  % today's date
\end{center}

\clearpage


 
\newpage
\section*{Declaration}

I, Paul Durbaba of Robinson College, being a candidate for Part II of the Computer
Science Tripos, hereby declare
that this dissertation and the work described in it are my own work,
unaided except as may be specified below, and that the dissertation
does not contain material that has already been used to any substantial
extent for a comparable purpose.

\bigskip
\leftline{Signed Paul Durbaba}

\medskip
\leftline{Date [date]}

\section*{Acknowledgements}

% TODO List the people that check the diss
LIST THE PEOPLE THAT CHECK THE DISS

%%%%%%%%%%%%%%%%%%%%%%%%%%%%%%%%%%%%%%%%%%%%%%%%%%%%%%%%%%%%%%%%%%%%%%%%%%%%%%
% Proforma, table of contents and list of figures

\setcounter{page}{1}
\pagenumbering{roman}
\pagestyle{plain}

\chapter*{Proforma}

{\large
	\begin{tabular}{ll}
		Name:               & \bf Paul Durbaba                       \\
		College:            & \bf Robinson College                     \\
		Project Title:      & \bf Compiling OCaml to WebAssembly \\
		Examination:        & \bf Part II Computer Science, May 2020        \\
		Word Count:         & \bf FILL IN LATER  \\
		Project Originator: & Tim Jones                \\
		Supervisor:         & Tobias Kohn            \\ 
	\end{tabular}
}
\footnotetext[1]{This word count was computed
	by {\tt detex diss.tex | tr -cd '0-9A-Za-z $\tt\backslash$n' | wc -w}
}
\stepcounter{footnote}


\section*{Original Aims of the Project}

At most 100 words describing the original aims of the project. 


\section*{Work Completed}

At most 100 words summarising the work completed. 

\section*{Special Difficulties}

At most 100 words describing any special difficulties that you faced.
(In most cases the special difficulties entry will say “None”.) 

\tableofcontents

\listoffigures

%%%%%%%%%%%%%%%%%%%%%%%%%%%%%%%%%%%%%%%%%%%%%%%%%%%%%%%%%%%%%%%%%%%%%%%
% now for the chapters

\clearpage        % just to make sure before the page numbering
                        % is changed

\setcounter{page}{1}
\pagenumbering{arabic}
\pagestyle{headings}

\chapter{Introduction}
% The Introduction should explain the principal motivation for the project. Show how the work fits into the broad area of surrounding Computer Science and give a brief survey of previous related work. It should generally be unnecessary to quote at length from technical papers or textbooks. If a simple bibliographic reference is insufficient, consign any lengthy quotation to an appendix. 

%TODO EXPLAIN KEY MOTIVATION, IDEAS BEHIND PROJECT


My motivation for this project is to learn how to make a compiler, and to learn both about programming in OCaml and about WebAssembly.

% TODO PREVIOUS RELATED WORK

\section{OCaml}
OCaml\cite{OCaml} is a strongly-typed functional programming language, with some imperative features such as references. I chose OCaml both as the source language of the compiler, and the language the compiler is designed in, because I wanted to gain some familiarity in writing programs in functional languages, and OCaml has similar syntax to Standard ML taught in first year, but with much better library support, and because compiling a functional programming language presents additional challenges to compiling an imperative language such as C - with first class functions and pattern matching requiring special consideration.

\section{WebAssembly}
% copied from project proposal
WebAssembly\cite{webassembly} is a stack-based binary instruction format for the web with the main goal of improving performance  of  more  computationally  intensive  functions  in  web  applications.   It  does not  replace  JavaScript  as  there  (currently)  is  no  way  to  perform  tasks  such  as  DOM manipulation  directly  from  WebAssembly  -  it  is  expected  that  a  JavaScript  application might  call  some  functions  implemented  in  WebAssembly  to  perform  computation,  and then display the results itself.
\\\\
WebAssembly was chosen as the target instruction set because it is relatively new, with few compilers out there currently targeting it, and it is likely to grow in popularity in the future as more extensions are added to it that make it more viable to be used - such as support for garbage collection and exceptions.
% TODO

%TODO WHAT IS WEBASSEMBLY


% TODO DESCRIPTION OF HOW TO BUILD THE PROJECT?

\section{Related Work}
% TODO Js\_of\_ocaml, A

There have been a few attempts to compile OCaml to WebAssembly already, such as by Sander Spies\cite{Awbfo}, who modified the existing backend of the OCaml Compiler to target WebAssembly. This differs from my approach in that I am implementing an entire compiler from the type-checker through to the WebAssembly code generator. While Sander's approach allows them to leverage the existing features and optimisations of the OCaml compiler, their approach didn't fit with my goals of learning how to write an entire compiler - including type checker, intermediate translations and optimisations by myself, and such an approach likely wouldn't constitute enough work for a Part II project.


\clearpage



\chapter{Preparation}
% Principally, this chapter should describe the work which was undertaken before code was written, hardware built or theories worked on. It should show how the project proposal was further refined and clarified, so that the Implementation stage could go smoothly rather than by trial and error.

% Throughout this chapter and indeed the whole dissertation, it is essential to demonstrate that a proper professional approach was employed.

% The nature of this chapter will vary greatly from one dissertation to another but, underlining the professional approach, this chapter will very likely include a section headed “Requirements Analysis” and incorporate other references to software engineering techniques.

% The chapter will cite any new programming languages and systems which had to be learnt and will mention complicated theories or algorithms which required understanding.

% It is essential to declare the Starting Point (see Section 7). This states any existing codebase or materials that your project builds on. The text here can commonly be identical to the text in your proposal, but it may enlarge on it or report variations. For instance, the true starting point may have turned out to be different from that declared in the proposal and such discrepancies must be explained. 

\section{Requirements}

The success criteria in the project proposal presents a clearly defined subset of OCaml to implement. This subset was designed to be large enough so that useful OCaml programs could be written in it, while small enough to be feasible to implement by Christmas.

% TODO STARTING POINT

% TODO MATERIAL DONE BEFORE CODE WAS WRITTEN

% TODO? HOW I ENSURED CODING WASN'T TRIAL AND ERROR

\section{Components of the Compiler}


\section{Libraries Used}


\section{Working Environment and Tools Setup}
I chose to use the Dune\cite{Dune} build system for OCaml as it is the most widely-used build system for OCaml, and supports multi-module projects and dependencies installed via OPAM, the OCaml Package Manager.
\\\\

\begin{minipage}{\linewidth}
Dune build files are specified in each directory a file called \textinline{dune}, with the top-level directory specifying the build file for the entire project, and subdirectories containing the build files for each module. These build files are specified in a LISP-like `s-expression' syntax, for instance here is the top-level build file of the project:
\begin{minted}{LISP}
(executable
    (name toplevel)
    (libraries proj.types proj.transform proj.codegen proj.base
               core_kernel compiler-libs.common)
    (preprocess (pps ppx_jane)))
\end{minted}
\end{minipage}
\\\\
The name field includes the `public names' of libraries to be included. Library modules have their own build files that specify `library' instead of `executable', and an additional `public\_name' field.
\\\\
The executable can be build by invoking \textinline{dune build toplevel.exe} in the top-level directory, which will output the binary to \textinline{_build/default/toplevel.exe}.

% TODO Dune, Git

\section{Starting Point}
Below is the starting point as stated in my project proposal.
% TODO copy from project proposal

\subsection{OCaml and Compilers}
I have done very little programming in OCaml previously, with the exception of coming up with some code samples when considering which features this compiler should support. However I have some experience in programming in Standard ML, a similar language, from Part IA Foundations of Computer Science.
\\\\
I also have little previous experience in writing compilers. I have written basic lexers and parsers, and implemented a system for interpreting mathematical expressions, however I have no experience with a larger compiler that compiles to instructions instead of interpreting.

\subsection{WebAssembly and the Web}
I have no experience with WebAssembly with the exception of reading through some of the documentation in the weeks leading up to this proposal.
\\\\
I have a fair amount of experience with JavaScript and websites: I do not anticipate any issue with coming up with a suitable demonstration of the WASM produced by the compiler.

\clearpage
\chapter{Implementation}
% This chapter should describe what was actually produced: the programs which were written, the hardware which was built or the theory which was developed. Any design strategies that looked ahead to the testing stage might profitably be referred to (the professional approach again).

% Descriptions of programs may include fragments of high-level code but large chunks of code are usually best left to appendices or omitted altogether. Analogous advice applies to circuit diagrams.

% Draw attention to the parts of the work which are not your own. The Implementation Chapter should include a section labelled "Repository Overview". The repository overview should be around one page in length and should describe the high-level structure of the source code found in your source code Repository. It should describe whether the code was written from scratch or if it built on an existing project or tutorial. Making effective use of powerful tools and pre-existing code is often laudable, and will count to your credit if properly reported.

% It should not be necessary to give a day-by-day account of the progress of the work but major milestones may sometimes be highlighted with advantage. 

% TODO WORK DONE, ONE SECTION PER PART OF THE COMPILER

\section{Front End}

\section{Type Checker}

\section{Lambda Lifting / Closure Conversion}

\section{Intermediate Translation}

\section{Code Generation}

\section{Overview of the files}

\clearpage
\chapter{Evaluation}
% This is where Assessors will be looking for signs of success and for evidence of thorough and systematic evaluation as discussed in Section 8.3. Sample output, tables of timings and photographs of workstation screens, oscilloscope traces or circuit boards may be included. A graph that does not indicate confidence intervals will generally leave a professional scientist with a negative impression.

% As with code, voluminous examples of sample output are usually best left to appendices or omitted altogether.

% There are some obvious questions which this chapter will address. How many of the original goals were achieved? Were they proved to have been achieved? Did the program, hardware, or theory really work?

% Assessors are well aware that large programs will very likely include some residual bugs. It should always be possible to demonstrate that a program works in simple cases and it is instructive to demonstrate how close it is to working in a really ambitious case. 

\section{Success Criteria}

\section{End To End Tester}

\section{Benchmarks / Benchmark Driven Optimisations}

% TODO DONT FORGET ALL THE WORK IN MAKING THE CODE SHORTER

% TODO HOW IT WAS EVALUATED
% TODO TEST SYSTEM AND BENCHMARKS




\clearpage
\chapter{Conclusion}
% This chapter is likely to be very short and it may well refer back to the Introduction. It might properly explain how you would have planned the project if starting again with the benefit of hindsight. 

% TODO CONCLUDE THE DOCUMENT
% TODO IVE SUCCESSFULLY MET SUCCESS CRITERIA, SUMMARY OF EVERYTHING
% TODO WHAT LESSONS DID I LEARN
% TODO WHAT WORK COULD YOU DO IN THE FUTURE




\clearpage

%%%%%%%%%%%%%%%%%%%%%%%%%%%%%%%%%%%%%%%%%%%%%%%%%%%%%%%%%%%%%%%%%%%%%
% the bibliography

\addcontentsline{toc}{chapter}{Bibliography}
\printbibliography[title={Bibliography}]
\clearpage

%%%%%%%%%%%%%%%%%%%%%%%%%%%%%%%%%%%%%%%%%%%%%%%%%%%%%%%%%%%%%%%%%%%%%
% the appendices
\appendix
% Assessors like to see some sample code or example circuit diagrams, and appendices are the sensible places to include such items. Accordingly, software and hardware projects should incorporate appropriate appendices. Note that the 12,000 word limit does not include material in the appendices, but only in extremely unusual circumstances may appendices exceed 10-15 pages - if you feel that such unusual circumstances might apply to you you should ask your Director of Studies and Supervisor to apply to the Chairman of Examiners. It is quite in order to have no appendices. Appendices should appear between the bibliography and the project proposal. 

\chapter{The First Appendix}

Things in appendix A


\clearpage

\chapter{The Second Appendix}

Things in appendix B


\clearpage

\chapter{Project Proposal}
\clearpage

\thispagestyle{empty}
	
	%\rightline{\large{Paul Durbaba}}
	%\medskip
	%\rightline{\large{Robinson}}
	%\medskip
	%\rightline{\large{pd452}}
    \rightline{\large{Candidate 2376D}}
	
	\vfil
	
	\centerline{\large Computer Science Tripos: Part II Project Proposal}
	\vspace{0.4in}
	\centerline{\Large\bf Compiling OCaml to WebAssembly}
	\vspace{0.3in}
	\centerline{\large{Friday 18\textsuperscript{th} October, 2019}}
	
	\vfil
	
	{\bf Project Originator:} Based on a proposal by Timothy Jones
	
	\vspace{0.1in}
	
	{\bf Resources Required:} See attached Project Resource Form
	
	\vspace{0.5in}
	
	{\bf Project Supervisor:} Tobias Kohn
	
	\vspace{0.2in}
	
	{\bf Signature:}
	
	\vspace{0.5in}
	
	{\bf Director of Studies:}  Alan Mycroft
	
	\vspace{0.2in}
	
	{\bf Signature:}
	
	\vspace{0.5in}
	
	{\bf Overseers:} Pietro Lio' and Robert Mullins
	
	\vspace{0.2in}
	
	{\bf Signatures:} 
	
	\vfil
	\eject
	
	
	\section*{Introduction and Description of the Work}
	The aim of the project is to implement a compiler from a subset of OCaml to WebAssembly.
	\\\\
	WebAssembly\footnote{https://webassembly.org/} is a binary instruction format for the web with the main goal of improving performance of more computationally intensive functions in web applications. It does not replace JavaScript as there (currently) is no way to perform tasks such as DOM manipulation directly from WebAssembly - it is expected that a JavaScript application might call some functions implemented in WebAssembly to perform computation, and then display the results itself.
	\\\\
	My work will involve writing a compiler in OCaml and a runtime system in a language such as C. 
	\\\\
	The compiler will take an AST (abstract syntax tree) produced by the lexer and parser of the OCaml compiler\footnote{https://github.com/ocaml/ocaml} (ocaml-compiler-libs), performing type checking (I considered using the OCaml type checker, but getting that to work might be just as hard as writing my own type checker), and then transforming the typed AST through a series of intermediate representations (IRs). Each IR will eliminate some higher-level feature, for instance replacing first-class functions with closures. At the end, I will generate instructions to target WebAssembly. The WebAssembly Binary Toolkit\footnote{https://github.com/WebAssembly/wabt} will then be used to output a usable WebAssembly module.
	\\\\
	The runtime system will be needed to implement closures and memory allocation / access in a way that can be used by the code generated by my compiler.
	
	\section*{Starting Point}
	\subsection*{OCaml and Compilers}
	I have done very little programming in OCaml previously, with the exception of coming up with some code samples when considering which features this compiler should support. However I have some experience in programming in Standard ML, a similar language, from Part IA Foundations of Computer Science.
	\\\\
	I also have little previous experience in writing compilers. I have written basic lexers and parsers, and implemented a system for interpreting mathematical expressions, however I have no experience with a larger compiler that compiles to instructions instead of interpreting.
	
	\subsection*{WebAssembly and the Web}
	I have no experience with WebAssembly with the exception of reading through some of the documentation in the weeks leading up to this proposal.
	\\\\
	I have a fair amount of experience with JavaScript and websites: I do not anticipate any issue with coming up with a suitable demonstration of the WASM produced by the compiler.
	
	
	\section*{Substance and Structure of the Project}
	
	%KEY CONCEPTS, MAJOR WORK ITEMS, THEIR RELATIONS AND RELATIVE IMPORTANCE, DATA STRUCTURES AND ALGORITHMS
	
	%Key Concepts: Type Checking, WebAssembly, Runtime System, Closure
	%Major Work Items: Components
	%Relative importance: All important!
	
	%Data Structures: Trees, lots of trees!

	The main components of the compiler are the type checker, the intermediate representations, the WebAssembly code generator, and the runtime system. These will need to be implemented in the order listed, as each depends on the last, with the exception of the runtime system and code generator which can be developed in parallel, as a change in one might require a change in the other. Likewise, they are all equally important in a complete compiler, as each part is required during the translation from AST through to a WebAssembly module.
	\\\\
	My subset of OCaml will include the following:
	\begin{itemize}
		\item Values: \camlinline{int} (64-bit signed integer), \camlinline{bool} and \camlinline{float} (64-bit floating point), able to be defined using expressions \camlinline{let} and \camlinline{let ... in}.
		\item Functions: \camlinline{let} and \camlinline{let rec} with multiple curried arguments, with an expression for the function body, as well as inline functions using \camlinline{fun}
		\item Types: Non-polymorphic types defined using \camlinline{type}, and tuples. These can be used to implement list types, and hence lists themselves will not be part of the subset (although supporting list syntax will be an extension)
		\item \camlinline{if ... then ... else} and basic pattern matching. I was planning on excluding pattern matching from the initial subset, however it is needed to match types and de-structure them, and alternate approaches would lead to clumsy syntax. Pattern matching will be limited to use on types.
	\end{itemize}
	Polymorphism is excluded from the initial subset as I do know how challenging to implement it will be, however it will likely be the first stretch goal that I try and implement.

	\subsection*{Testing}
	I intend to write unit tests, for instance using the OUnit\footnote{https://github.com/gildor478/ounit} unit testing framework. These will function to demonstrate that sub-components are working before the whole compiler can be tested, and more importantly to prevent the re-introduction of bugs that have already been fixed by writing a unit test when a bug is found to ensure it doesn't occur again.
	\\\\
	The overall compiler will be tested with a selection of code samples designed to test important features and edge cases. The compiled code will be executed in the browser and the result checked against the expected result. This could also be automated, e.g. by using a WebAssembly interpreter such as the WebAssembly reference interpreter\footnote{https://github.com/WebAssembly/spec/tree/master/interpreter}, depending on the complexity of this.
	
	\subsection*{Evaluation}
	I will evaluate my compiler in comparison to other methods of executing OCaml, particularly on the web, such as the OCaml compiler and Js\_of\_ocaml\footnote{https://ocsigen.org/js\_of\_ocaml/3.1.0/manual/overview}. I will do this by measuring performance in terms of how long particular samples of OCaml code take to execute on these different platforms, by running them multiple times on each platform (or the sample includes a loop of the same effect), and recording the time between execution starting and execution completing.
	\\\\
	It would also be interesting to compare memory usage between the different approaches, however that will likely be unfeasible due to the difficulty of getting accurate memory usage statistics, e.g. how much of the browser's memory is used by the JavaScript OCaml interpreter.
	\\\\
	When optimisations are implemented, I will analyse the performance with and without the optimisation, for instance by using timing facilities available in JavaScript. An optimisation will be considered successful if it leads to a measurable increase in execution speed (or a measurable decrease in memory usage) for a carefully chosen set of OCaml programs.
	
	
	\section*{Success Criterion}
	The project will be a success if the following criteria are met:
	\begin{itemize}
		\item The compiler can take a selection of reasonable (not likely to result in more than 10MB of memory allocation) samples written using all features of the selected subset of OCaml, and compile them to WebAssembly modules
		\item These WebAssembly modules can be loaded and executed with a set of provided input values for each sample in the latest version of Chrome and Firefox to produce the same results as if the OCaml code they were compiled from was compiled, executed, and given the same input value by the OCaml Compiler
	\end{itemize}
	
	
	\section*{Possible Extensions}
	% Pattern Matching, References and Loops, Garbage Collection, Exceptions,
	\begin{itemize}
		\item Expanding my subset of OCaml supported: polymorphism, more advanced pattern matching, list syntax, strings, references, for and while loops, modules...
		
		\item Improving code access from JavaScript, for instance allowing JavaScript objects (probably just numbers and strings) to be passed to OCaml functions, likely with a special type to represent JavaScript objects
		
		\item Optimisations: There are many optimisations that could be implemented to improve the code generated by the compiler, for instance peephole optimisation, dead-code elimination, constant propagation, function call inlining, and many more that I could learn about.
		
		\item End to End testing: Using a WebAssembly interpreter, whereby I can specify a sample of programs to compile, and the expected output from running them, and validate that they all compile and produce the expected results in the interpreter.
		
		\item Supporting Exceptions. This would be challenging due to WebAssembly's current lack of support for them, however there are ways of working around this, with their own performance costs. One way might be to make functions that throw exceptions return a type that signifies either a result or an exception was thrown, then we can check if an exception was thrown after each function call, and propagate exceptions until a relevant catch block is found.
		
		\item Garbage Collection. WebAssembly likewise does not currently support garbage collection, and additionally does not allow walking the stack for security reasons. There are limited ways to work around this: by storing all values except for primitive types and references on the heap, reference counting could be used to track which heap items have references on the stack, and tracing could be used between heap items. A system like this would likely have to be implemented in JavaScript or TypeScript due to the complexity of getting it right in WebAssembly.
		
	\end{itemize}
	
	\section*{Timetable and Milestones}
	
	\subsection*{25th Oct - 8th Nov}
	Getting ready:
	\begin{itemize}
		\item Creating a git repository and GitHub remote for the project, and skeleton for the overall project
		\item Setting up the required libraries (the ocaml-compiler-libs)
		\item Setting up and trying out unit testing with OUnit
		\item Parsing OCaml files using the ocaml-compiler-libs parser, and experimenting with the resulting ASTs to learn more about them 
	\end{itemize}
	In addition, I will practice coding in OCaml to learn more about the language and come up with useful code samples that will be useful for testing later on. The book \textit{Types and Programming Languages} by Benjamin C. Pierce contains some OCaml samples to help me learn and also prepare me for writing the type checker.
	
	\subsection*{9th Nov - 22nd Nov}
	I will implement a type checker for the initial subset of OCaml. If I have time, polymorphism (an extension) could also be implemented at this stage, certainly my implementation of type checking will have to consider how polymorphism will be implemented in the future.
	\\\\
	At the end of this I should have a function that can take an OCaml AST and output a typed AST, with a type attached to each node.
	
	\subsection*{23nd Nov - 6 Dec}
	Implement translation of the typed AST through to a lower-level representation suitable for transformation into WebAssembly instructions.
	
	\subsection*{7 Dec - 20 Dec}
	Implement code generation and a suitable runtime system for WebAssembly to perform memory allocation and closure calls, probably using C. After this point, it should be possible to use the compiler to compile some code samples and execute them in a browser, and hence I will have met my success criterion.
	
	\subsection*{20 Dec - 3 Jan}
	I will probably be taking about a week's break from the project, and using the remaining time in these two weeks to clean up / improve the project if the success criterion has been met (for instance by refactoring, implementing additional test cases or adding better error messages), and continue working up to the success criterion otherwise.
	
	\subsection*{4 Jan - 17 Jan}
	I will implement polymorphism in the type checker and some form of polymorphism elimination in the intermediate representation. I also hope to implement references and for/while loops, which should be fairly simple to implement but will allow more interesting programs to be compiled.
	
	\subsection*{17 Jan - 31 Jan}
	I will implement some optimisations such as the ones I listed above, and evaluate them as I described in the evaluation section.
	\\\\
	Write the progress report.
	
	\subsection*{1 Feb - 14 Feb}
	Presentations and preparation needed for them.
	\\\\
	I will investigate and implement some way of passing JavaScript values (numbers and strings) to functions, and seek to improve my pattern matching implementation by compiling more complicated match statements.
	
	\subsection*{15 Feb - 28 Feb}
	I will set up an end to end testing system as described in the stretch goals section.
	
	\subsection*{1 Mar - 13 Mar}
	Start writing dissertation. Investigate supporting exceptions and garbage collection - perhaps the WebAssembly extensions to support them will be finished by this time. I will decide to work on either exceptions, garbage collection or additional optimisations, depending on the complexity of supporting exceptions or garbage collection.
	
	\subsection*{14 Mar - 27 Mar}
	Continue writing dissertation. Attempt to wrap up remaining unfinished features e.g. those I started in the last two weeks.
	
	\subsection*{28 Mar - 10 Apr}
	Continue with the dissertation, submitting first drafts of chapters to my supervisor. Focus on improving the code e.g. by refactoring, or implementing additional tests.
	
	\subsection*{10 Apr - 24 Apr}
	Fully focus on the dissertation, leaving the code as it is. Submit draft version of dissertation to supervisor and DoS.
	
	\subsection*{24th Apr - 8 May}
	At this point the dissertation should be complete, with only minor changes to be made over these weeks.
	\\\\
	Towards the end of these two weeks, I will submit the final version of my dissertation.
	
	
	\section*{Resource Declaration}
	I plan to do the work via remote desktop to my server located in France, so I can easily switch between using my desktop and laptop. This has 32GB RAM and two 2 TB HDDs in RAID 1 (mirrored). I will use git for version control of code and important documents, which will be regularly pushed to a private GitHub repository. In addition, both my laptop and desktop automatically download backups of data on my server at regular intervals when they are online.
	\\\\
	I accept full responsibility for this machine and I have made contingency plans to protect myself against hardware and/or software failure.
	\\\\
	I will be making use of some components of the OCaml compiler (the lexer, parser, and possibly type-checker), as well as the WebAssembly Binary Toolkit (WABT).

\end{document}
