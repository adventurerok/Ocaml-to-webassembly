% The master copy of this demo dissertation is held on my filespace
% on the cl file serve (/homes/mr/teaching/demodissert/)

% Last updated by MR on 2 August 2001

\documentclass[12pt,twoside,notitlepage]{report}

\usepackage{a4}
\usepackage{verbatim}

\usepackage{amsmath}
\usepackage{amssymb}
\usepackage{mathtools}

\usepackage{minted}
%\usemintedstyle{colorful}
\setmintedinline{breaklines}

\newcommand{\textinline}{\mintinline{text}}
\newcommand{\cinline}{\mintinline{C}}
\newcommand{\camlinline}{\mintinline{OCaml}}

\newcommand\note[1]{\textcolor{blue}{#1}}

\input{epsf}                            % to allow postscript inclusions
% On thor and CUS read top of file:
%     /opt/TeX/lib/texmf/tex/dvips/epsf.sty
% On CL machines read:
%     /usr/lib/tex/macros/dvips/epsf.tex



\raggedbottom                           % try to avoid widows and orphans
\sloppy
\clubpenalty1000%
\widowpenalty1000%

\addtolength{\oddsidemargin}{6mm}       % adjust margins
\addtolength{\evensidemargin}{-8mm}

\renewcommand{\baselinestretch}{1.1}    % adjust line spacing to make
                                        % more readable

\usepackage[backend=bibtex, style=alphabetic, sorting=ynt]{biblatex}
\addbibresource{refs.bib}

\begin{document}



%%%%%%%%%%%%%%%%%%%%%%%%%%%%%%%%%%%%%%%%%%%%%%%%%%%%%%%%%%%%%%%%%%%%%%%%
% Title


\pagestyle{empty}

\hfill{\LARGE \bf Paul Durbaba}

\vspace*{60mm}
\begin{center}
\Huge
{\bf Compiling OCaml to WebAssembly} \\
\vspace*{5mm}
Diploma in Computer Science \\
\vspace*{5mm}
Robinson College \\
\vspace*{5mm}
May 2020  % today's date
\end{center}

\clearpage


 
\newpage
\section*{Declaration}

I, Paul Durbaba of Robinson College, being a candidate for Part II of the Computer
Science Tripos, hereby declare
that this dissertation and the work described in it are my own work,
unaided except as may be specified below, and that the dissertation
does not contain material that has already been used to any substantial
extent for a comparable purpose.

\bigskip
\leftline{Signed Paul Durbaba}

\medskip
\leftline{Date [date]}

\section*{Acknowledgements}

% TODO List the people that check the diss
\note{LIST THE PEOPLE THAT CHECK THE DISS}

%%%%%%%%%%%%%%%%%%%%%%%%%%%%%%%%%%%%%%%%%%%%%%%%%%%%%%%%%%%%%%%%%%%%%%%%%%%%%%
% Proforma, table of contents and list of figures

\setcounter{page}{1}
\pagenumbering{roman}
\pagestyle{plain}

\chapter*{Proforma}

{\large
	\begin{tabular}{ll}
		Name:               & \bf Paul Durbaba                       \\
		College:            & \bf Robinson College                     \\
		Project Title:      & \bf Compiling OCaml to WebAssembly \\
		Examination:        & \bf Part II Computer Science, May 2020        \\
		Word Count:         & \bf FILL IN LATER  \\
		Project Originator: & Timothy M. Jones                \\
		Supervisor:         & Tobias Kohn            \\ 
	\end{tabular}
}
%\footnotetext[1]{This word count was computed
%	by {\tt detex diss.tex | tr -cd '0-9A-Za-z $\tt\backslash$n' | wc -w}
%}
%\stepcounter{footnote}


\section*{Original Aims of the Project}

\note{At most 100 words describing the original aims of the project.}


\section*{Work Completed}

\note{At most 100 words summarising the work completed.}

\section*{Special Difficulties}

\note{At most 100 words describing any special difficulties that you faced.
(In most cases the special difficulties entry will say “None”.) }

\tableofcontents

\listoffigures

%%%%%%%%%%%%%%%%%%%%%%%%%%%%%%%%%%%%%%%%%%%%%%%%%%%%%%%%%%%%%%%%%%%%%%%
% now for the chapters

\clearpage        % just to make sure before the page numbering
                        % is changed

\setcounter{page}{1}
\pagenumbering{arabic}
\pagestyle{headings}

\chapter{Introduction}
\note{The Introduction should explain the principal motivation for the project. Show how the work fits into the broad area of surrounding Computer Science and give a brief survey of previous related work. It should generally be unnecessary to quote at length from technical papers or textbooks. If a simple bibliographic reference is insufficient, consign any lengthy quotation to an appendix.}
%TODO EXPLAIN KEY MOTIVATION, IDEAS BEHIND PROJECT
My motivation for this project is to learn how to make a compiler. Compilers are essential to computing because they enable the transformation of code from high-level languages that are intuitive for use by us, into low-level machine understandable code than can actually be executed, and writing a compiler is a substantial software project that invokes many areas of computer science such as type theory and program analysis.

% TODO PREVIOUS RELATED WORK

\section{OCaml}
OCaml\cite{OCaml} is a strongly-typed functional programming language, with some imperative features such as references. I chose OCaml both as the source language of the compiler, and the language the compiler is designed in, because I wanted to gain some familiarity in writing programs in functional languages, and OCaml has similar syntax to Standard ML taught in first year, but with much better library support, and because compiling a functional programming language presents additional challenges to compiling an imperative language such as C - with first class functions and pattern matching requiring special consideration.

\section{WebAssembly}
% copied from project proposal
WebAssembly\cite{webassembly} is a stack-based binary instruction format for the web with the main goal of improving performance  of  more  computationally  intensive  functions  in  web  applications.  It  does not  replace  JavaScript  as  there  (currently)  is  no  way  to  perform  tasks  such  as  DOM manipulation  directly  from  WebAssembly  ---  it  is  expected  that  a  JavaScript  application might  call  some  functions  implemented  in  WebAssembly  to  perform  computation,  and then display the results itself.
\\\\
The current MVP (minimum viable product) version of WebAssembly is designed for compiling languages like C and C++ that do not use garbage collection and can make do without exceptions. There are extensions currently being developed to add support for these features, but progress on these is slow as they often depend on other extensions, for instance the garbage collection extension depends on extensions for reference types and typed function references, which seek to expand the WebAssembly type system so for instance a garbage collector would understand the shape of data in memory.\cite{Wgce}
\\\\
WebAssembly was chosen as the target instruction set because it is relatively new, with few compilers out there currently targeting it, and it is likely to grow in popularity in the future as more extensions are added to it that make it more viable to be used - such as support for garbage collection and exceptions.
% TODO

%TODO WHAT IS WEBASSEMBLY


% TODO DESCRIPTION OF HOW TO BUILD THE PROJECT?

\section{Related Work}
% TODO Js\_of\_ocaml, A

There have been a few attempts to compile OCaml to WebAssembly already, such as by @SanderSpies\cite{Awbfo}, who modified the existing backend of the OCaml Compiler to target WebAssembly. Their attempt worked from the `CMM' --- the final stage of the OCaml compiler before code generation, modifying that to include extra type information for WebAssembly, and then doing code generation. This differs from my approach in that I am implementing almost an entire compiler from the type-checker through to the WebAssembly code generator, but excluding lexing/parsing.
\\\\
While Sander's approach allows them to leverage the existing features and optimisations of the OCaml compiler, their approach didn't fit with my goals of learning how to write an entire compiler - including type checker, intermediate translations and optimisations by myself, and such an approach likely wouldn't constitute enough work for a Part II project.


\clearpage



\chapter{Preparation}
\note{Principally, this chapter should describe the work which was undertaken before code was written, hardware built or theories worked on. It should show how the project proposal was further refined and clarified, so that the Implementation stage could go smoothly rather than by trial and error.}

\note{Throughout this chapter and indeed the whole dissertation, it is essential to demonstrate that a proper professional approach was employed.}

\note{The nature of this chapter will vary greatly from one dissertation to another but, underlining the professional approach, this chapter will very likely include a section headed “Requirements Analysis” and incorporate other references to software engineering techniques.}

\note{The chapter will cite any new programming languages and systems which had to be learnt and will mention complicated theories or algorithms which required understanding.}

\note{It is essential to declare the Starting Point (see Section 7). This states any existing codebase or materials that your project builds on. The text here can commonly be identical to the text in your proposal, but it may enlarge on it or report variations. For instance, the true starting point may have turned out to be different from that declared in the proposal and such discrepancies must be explained. }

\section{Requirements}

The success criteria in the project proposal presents a clearly defined subset of OCaml to implement. This subset was designed to be large enough so that useful OCaml programs could be written in it, while small enough to be feasible to implement by Christmas.

% TODO STARTING POINT

% TODO MATERIAL DONE BEFORE CODE WAS WRITTEN

% TODO? HOW I ENSURED CODING WASN'T TRIAL AND ERROR

\section{Components of the Compiler}


\section{Working Environment and Tools Setup}
Since I have both a laptop and desktop, I decided the best way to work on both would be to do the work remotely via a remote desktop application on a remote server. Both my laptop and desktop were configured to download backups from the server once per hour (if they were on), so if there proved to be a problem with the server, I could redeploy easily by uploading the backups to a new server if required.
\\\\
In addition, I used Git in order to keep a record of my work, to allow me to access previous versions of files, and to backup to GitHub

\subsection{Dune}
I chose to use the Dune\cite{Dune} build system for OCaml as it is the most widely-used build system for OCaml, and supports multi-module projects and dependencies installed via OPAM, the OCaml Package Manager.
\\\\
Dune build files are specified in each directory a file called \textinline{dune}, with the top-level directory specifying the build file for the entire project, and subdirectories containing the build files for each module. These build files are specified in a LISP-like `s-expression' syntax, for instance here is the top-level build file of the project:
\\\\
\begin{minipage}{\linewidth}

\begin{minted}{LISP}
(executable
    (name toplevel)
    (libraries proj.types proj.transform proj.codegen proj.base
               core_kernel compiler-libs.common)
    (preprocess (pps ppx_jane)))
\end{minted}
\end{minipage}
\\\\
The name field includes the `public names' of libraries to be included. Library modules have their own build files that specify `library' instead of `executable', and an additional `public\_name' field.
\\\\
The executable can be build by invoking \textinline{dune build toplevel.exe} in the top-level directory, which will output the binary to \textinline{_build/default/toplevel.exe}.

% TODO Dune, Git

\section{Libraries / Tools Used}
\subsection{OCaml Compiler Libs}
I used the official OCaml compiler libraries to perform lexing and parsing, which provide the same frontend as used by the official OCaml compiler. As the frontend of the official OCaml compiler is liable to change between releases, I stuck to version 4.08, which was the most recent version when I started the project but has now been succeeded by 4.09.
\note{Citation needed}

\subsection{Web Assembly Binary Toolkit}
The Web Assembly Binary Toolkit is a separate tool which can compile WebAssembly Text Form (.wast files) outputted by my compiler to the WebAssembly Binary Format (.wasm files) that can be loaded by NodeJS/browsers.
\note{Citation needed}

\section{Starting Point}
I had very little OCaml experience prior to this project, and no recent experience of writing compilers besides from writing a mathematical expression interpreter a few years ago. In addition, I had no experience of using WebAssembly, but I have plenty of experience writing JavaScript.
\\\\
I attempted to deal with some of these issues prior to starting the project by coming up with some OCaml samples for the compiler to compile in the future, and by setting up an OCaml workspace where I successfully figured out how to import the OCaml Compiler libraries, and learned to navigate the AST they use by writing a simple example that adds one to integer constants. I also read through the WebAssembly documentation to get a sense of which features would be a challenge to compile to WebAssembly.

\clearpage
\chapter{Implementation}
\note{This chapter should describe what was actually produced: the programs which were written, the hardware which was built or the theory which was developed. Any design strategies that looked ahead to the testing stage might profitably be referred to (the professional approach again).}

\note{Descriptions of programs may include fragments of high-level code but large chunks of code are usually best left to appendices or omitted altogether. Analogous advice applies to circuit diagrams.}

\note{Draw attention to the parts of the work which are not your own. The Implementation Chapter should include a section labelled "Repository Overview". The repository overview should be around one page in length and should describe the high-level structure of the source code found in your source code Repository. It should describe whether the code was written from scratch or if it built on an existing project or tutorial. Making effective use of powerful tools and pre-existing code is often laudable, and will count to your credit if properly reported.}

\note{It should not be necessary to give a day-by-day account of the progress of the work but major milestones may sometimes be highlighted with advantage. }

% TODO WORK DONE, ONE SECTION PER PART OF THE COMPILER



\section{Front End}
Lexing and parsing is handled by the OCaml Compiler Libs, which produces a \textinline{structure} object representing the AST of the entire program. This AST and the corresponding typed-AST, which I designed to be very similar (using the same names prefixed with a t), are used throughout the first half of the compilation process before intermediate translation occurs, and as such, I will give an overview of their components:
\begin{itemize}
	\item \textinline{(t)structure} A list of \textinline{(t)structure_item} representing the top-level items in a file.
	\item \textinline{(t)structure_item} A single top-level item in a file. Can either be an \textinline{(t)expression} (e.g. for imperative code), a let binding consisting of a list of \textinline{(t)value_binding}, or a type definition consisting of a list of \textinline{type_decls}
	\item \textinline{(t)expression} A unit of code that will produce a value. There are various types including identifiers, constants, match statements, function definitions and applications.
	\item \textinline{(t)pattern} A structure of a value, that can be used both to test if a value conforms to this structure (e.g. in a match statement), and to destructure a value into separate variables (e.g. in a let binding). Can be an identifer, constant, tuple of several patterns, or a construct (effectively a named tuple).
	\item \textinline{(t)value_binding} An expression and a pattern. The result of evaluating the expression is destructured and bound to the variables in the pattern, for instance in \camlinline{let x,y = (5 + 2, 3)} the pattern is \camlinline{(x, y)} and \camlinline{(5 + 2, 3)} is the expression
\end{itemize}
\note{Put the typed AST and the IR in an appendix?}





\section{Type Checker}
\note{
	Does multiple things
	\begin{itemize}
		\item HM type inference, generating constraints and then unifying them
		\item Builds up information about type definitions e.g. construct types and their constructors
		\item Makes the typed-ast
		\begin{itemize}
			\item Small differences to make the typed-ast nicer to work with
			\item Variables get additional unique ID
			\item Patterns get a list of variables they define and their types
			\item Value bindings get a similar list with the generalized types
		\end{itemize}
	\end{itemize}
}
The type-checker is responsible for translating the untyped AST into a typed-AST, and in doing so ensuring that the program is well typed, inferring types that are unknown. The main algorithm that I use is a inference algorithm based on the Hindley-Milner type system. This is a constraint based algorithm that builds up a set of constraints about types at different points in the program, and solves them to produce a unification mapping type-variables to types. My implementation was guided by an article about using Hindley-Milner in Haskell\cite{Hmi}.
\\\\
The definitions of the types I used are shown in Figure \ref{fig:types}. A \camlinline{scheme} is a special kind of type which I explain with let-bindings. Type constraints are implemented as \camlinline{scheme_type} pairs.
\begin{figure}
	\begin{minted}[linenos]{OCaml}
	type tvalue = V_unit | V_int | V_bool | V_float
	
	type scheme_type =
	| T_var of string (* Type variable *)
	| T_val of tvalue
	| T_tuple of scheme_type list
	| T_constr of string * scheme_type list (* Constructs *)
	| T_func of scheme_type * scheme_type (* Function type *)
	
	type scheme = Forall of String.Set.t * scheme_type
	\end{minted}
	\caption{The definition of types}
	\label{fig:types}
\end{figure}
\\\\
The main bulk of the algorithm is generating constraints for expressions and patterns. This is implemented as a depth-first traversal of the AST, generating types and constraints for the `leaves' of the tree and then using rules to generate these constraints for internal nodes in the tree. Some of these rules are shown in figure \ref{fig:typerules}, which uses type rules of the form $\Gamma \vdash e : t, C$, where $\Gamma$ is the type-context (stores the mapping of variables to types), e is the expression or pattern being typed, t is the type, and C is the list of constraints (with $@$ used to append lists). $ftv(\Gamma)$ is the set of free type variables in $\Gamma$.
\begin{figure}
	$$\begin{array}{cc}
	\dfrac{x : f \in \Gamma \quad t = \text{instantiate(f)}}{\Gamma \vdash x : t, \{\}} \textsc{IdentExpr} &
	\dfrac{a \notin ftv(\Gamma)}{\Gamma, x : a \vdash x : a, \{\}} \textsc{IdentPat}
	\end{array}$$
	$$\dfrac{\Gamma,\Gamma_1 \vdash p : t_1,\ C_1 \quad \Gamma, \Gamma_1 \vdash e : t_2,\ C_2}{\Gamma \vdash (\text{fun}\ p \rightarrow e) : t_1 \rightarrow t_2,\ C_1\ @\ C_2} \textsc{Fun}$$
	$$\dfrac{\Gamma \vdash x : t_1,\ C_1 \qquad \Gamma \vdash y : t_2,\ C_2 \quad a \notin ftv(\Gamma)}{\Gamma \vdash x\ y : a,\ C_1\ @\  C_2 \ @\  [t_1 = t_2 \rightarrow a]} \textsc{Apply}$$
	$$\dfrac{n \in \mathbb{N}}{\Gamma \vdash n : \text{int}, \{\}} \textsc{ConstantInt}$$
	$$\Gamma, \Gamma_1 \vdash p : t_1, C_1 \quad \Gamma \vdash e_1 : t_2, C_2 \quad C_3 = C_1\ @\ C_2\ @\ [t_1 = t_2] \quad s = \text{solve}(C_3)$$
	$$\text{For each variable added by p, we now substitute s into its type and generalize it over }$$
	$$ \Gamma \text{ to produce } \Gamma_2, \text{ which is } \Gamma_1 \text{ with the generalizations}$$
	$$\text{We extract outer constraints from } C_3 \text{ to produce } C_4$$
	$$\dfrac{\Gamma, \Gamma_2 \vdash e_2 : t_3, C_5}{\Gamma \vdash \text{let}\ p = e_1\ \text{in}\ e_2 : t_3,\ C_4\ @\ C_5} \textsc{Let}$$
	\caption{Some rules of my Hindley-Milner type system\note{Need to explain, also they are very complicated even before I introduce instantiation and generalization with let bindings. Also, inference for patterns takes an input context and produces an output context. A pattern adds a variable to a context, while a let binding would make that forall over it's free types (generalization). Best to make a multi-line let rule with an explanation of each part? And it's possible to have type variables that are not in ftv(Gamma) that are in use}}
	\label{fig:typerules}
\end{figure}





\section{Lambda Lifting / Closure Conversion}
%\note{Replace function definitions with mk\_closure, extracting function to it's own object. Track the free variables inside the function}

Lambda Lifting and Closure Conversion is the process whereby function definitions are extracted from the AST, replacing the original definition site with a special operation that constructs a closure of the original function, passing in the required environment variables as needed.
\begin{figure}[h]
	\begin{minted}[linenos]{OCaml}
(* Curried function definition *)
let sum (x, y) (z, w) = x + y + z + w

(* Curried function representation in AST *)
let sum = (fun (x, y) -> (fun (z, w) -> x + y + z + w))

(* Modified AST and extracted functions *)
let sum = mk_closure $$f_sum ()
$$f_sum (): fun arg_sum -> mk_closure $$f_sum-app (arg_sum)
$$f_sum-app (arg_sum): fun arg_sum-app ->
let (x, y) = arg_sum in
let (z, w) = arg_sum-app in
x + y + z + w
	\end{minted}
	\caption{Curried functions in closure conversion}
	\label{fig:curried}
\end{figure}
\\\\
This is achieved by walking the typed-AST. Each time a function definition is encountered, we must construct an extracted function definition, and modify the AST to include a make-closure operation. This is done in the following order, with figure \ref{fig:curried} used to illustrate:
\begin{enumerate}
	\item We give a unique name the function. There are three possible cases here:
	\begin{itemize}
		\item The function is defined in a let binding (e.g. \camlinline{sum}), in which case we use the name of the variable in the let binding.
		\item The function is defined as the expression of another function. This occurs in curried functions which are represented in the AST as nested function definitions. In this case we take the name of the parent function, and append `-app'. (e.g. \camlinline{sum-app})
		\item The function is defined anonymously, in which case we give it an anonymous name.
	\end{itemize}
	\item For curried functions, we create a new argument (single variable pattern) for each function (e.g. \camlinline{arg-sum}), and put let expressions to bind these arguments to the original patterns inside the body of the innermost function in the curried definition. This ensures that we only need to store one environment variable per curried argument inside the closures of deeper functions. For instance, \camlinline{arg-sum} is stored inside the closure for \camlinline{$$f_sum-app}, instead of needing to store both \camlinline{x} and \camlinline{y}.
	
	\item We recursively apply closure conversion to the expression of the function, to give us the expression to use in the extracted function definition.
	\item We perform a depth-first search of this expression to determine the free variables, which will become `closure arguments' for the extracted function. For instance, \camlinline{arg-sum} is free inside the body of \camlinline{$$f_sum-app}, so it must become a closure argument for this extracted function.
	\item We replace the function definition with a `special' make closure operation, which takes the name of the extracted function and the closure arguments as parameters. \camlinline{mkclosure $$f_sum-app (arg_sum)}
	\item We create an object to represent the extracted function, which contains it's name, argument pattern, expression, and list of closure variables and their types.
\end{enumerate}
The modified top-level typed-AST, along with these extracted function definitions, are then used as the representation of the program until translation into the intermediate representation occurs.





\section{Optimisations on Closure Converted Typed AST}
I then perform two optimisations on this closure-converted typed-AST:
\begin{itemize}
	\item Direct Call Generation replaces applications of closures with direct calls of the corresponding function, if we can be sure the closure is always one of that function. This is the most effective optimisation I implement because it eliminates the overhead of closure calls, which involves wrapper functions to load the closure arguments from the closure before calling the actual function, and it can be applied to curried functions, enabling multiple arguments to be passed at once without constructing the intermediate closures.
	\item Tail Call Optimisation. A function is tail-recursive if it is recursive (calls itself), but only calls itself as the very last operation to return a result, and never performs additional computation after calling itself. Tail-recursive functions can be modified to a loop instead of calling themselves, which avoids the function call overhead of putting an additional frame on the stack.
\end{itemize}
Both of these are implemented through rule-based analysis of abstract values.

\subsection{Direct Call Generation}
Direct Call Generation walks the AST, diving into functions at the location of their make closure operation, determining which expressions and variables at different point are certainly closures for a specific function, and which variables are available at each point. When a closure application is encountered, we do the following:
\begin{enumerate}
	\item Check if the value being applied is known to be a closure to a specific function.
	\item Check if that function's closure variables are available. Because we do not rename closure variables inside functions, it suffices that if a variable e.g. \textinline{x#3} is available outside the closure, and is also a closure variable, then they refer to the same value.
	\item If we have a single application, we can replace directly with a call that provides the necessary closure variables.
	\item If we have a curried application, we must first determine the order to pass the arguments, and then we can pass all the arguments at once along with the closure variables.
\end{enumerate}

\subsection{Tail Call Optimisation}
Before we can apply tail call optimisation to a function, we must first check if it is tail-recursive. To do this we use a rule-based analysis with three possible `types': simple, tail-recursive and recursive. A simple function returns a value without calling itself, a tail-recursive function calls itself only as the last thing it does before it returns, and a recursive function calls itself and then performs additional computation afterwards.
\\\\
The rules are quite simple:
\begin{itemize}
	\item Constants and calls to other functions have type \textinline{text} providing all the arguments are \textinline{simple}
	\item Recursive calls to the same function have type \textinline{tailrec} providing all the arguments are \textinline{simple}.
	\item An expression that performs computation on its subexpressions (e.g. addition) has type \textinline{simple} if all it's arguments are \textinline{simple}, otherwise it is \textinline{recursive}.
	\item An expression that doesn't perform computation gives the worst type of it's subexpressions.
\end{itemize}
For instance, \camlinline{if simple then simple else tailrec} has type \textinline{tailrec} because it does not perform computation on the result of the \textinline{tailrec}, but \camlinline{if tailrec then simple else tailrec} has type \textinline{recursive} because it uses the result of the condition to determine which branch to choose.
\\\\
Once we have discovered a tail-recursive function, we then convert it to use a while loop as follows. Figure \ref{fig:tailrec} shows this in action:
\begin{enumerate}
	\item We create references for each argument, and for the result
	\item Inside the body of the while loop, we start by loading the arguments from the references
	\item We replace each tail call with loading the correct arguments into the references, and starting the loop body again (`continuing')
	\item If we manage to compute the result, we store it inside the result reference
	\item Once outside the loop, we simply dereference the result reference
\end{enumerate}
This gives us our iterative function body. It can also introduce `broken, unreachable' code, for instance the \camlinline{continue} statement is of type unit so it must return a unit type which is then assigned to result which is not of type unit. This code is clearly broken, but it is also unreachable as the \camlinline{continue} statement restarts the loop instead, so unreachable code analysis is required to eliminate it later. Equally broken is the creation of the result reference if our function does not return an integer, however this can be eliminated via dead-code elimination.

\begin{figure}[h]
\begin{minipage}{0.5\linewidth}
	\begin{minted}[linenos]{OCaml}
let rec fact n acc =
  if n = 0 then
    acc
  else
    fact (n - 1) (n * acc)
	\end{minted}
\end{minipage}
\begin{minipage}{0.5\linewidth}
\begin{minted}[linenos]{OCaml}
let fact n_in acc_in =
  let n_ref = ref n_in in
  let acc_ref = ref acc_in in
  let result_ref = ref 0 in
  while true do
    let n = !n_ref in
    let acc = !acc_ref in
    let result =
      if n = 0 then
        acc
      else
        (n_ref := n - 1;
        acc_ref := n * acc;
        continue)
    in
    result_ref := result
  done
  !result_ref
\end{minted}
\end{minipage}
	\caption{Before and after of tail-recursion optimisation}
	\label{fig:tailrec}
\end{figure}





\section{Intermediate Translation}
%\note{
%	What are the interesting cases?
%	\begin{itemize}
%		\item Mutual recursion: Create closures and then fill them up
%		\item Patterns: Generate code to both check them and destructure
%		\item Match Statements: Go through each block, exit the block early if match fails, until we finish a block and then can exit the whole match
%		\item Boxing of floats
%	\end{itemize}
%}

The original intermediate representation of the compiler was designed to target WebAssembly as simply as possible, and as such was structured and stack-based, with instructions pushing and popping values to the stack, and even containing lists of sub-instructions to evaluate arguments. While this approach allowed me to reach my success criteria quickly, it would have been very difficult to optimise for due to the need to keep track of what data is on the stack while doing analyses.
\\\\
Thus I introduced a new `variable-based' unstructured intermediate representation, where each instruction can take multiple variables as arguments and write a result to one variable, and structure is represented by special begin and end instructions for each structure.
\\\\
The full IR is quite large, but its instructions can be broken down into one of four categories:
\begin{itemize}
	\item Basic operations on variables, e.g. assigning a constant to a variable, copying a variable, and unary and binary operations
	\item Control instructions: those that mark the start and end of blocks, loops and if-else statements, and instructions for jumping out of these structures, as well as the special `fail' instruction.
	\item Memory operations, e.g. creating and loading from tuples/constructs/boxes and closures
	\item A closure-calling instruction, and a direct-call instruction.
\end{itemize}
The types used in the Intermediate Representation are shown in \ref{fig:ir}, with the full IR available in Appendix B.
\begin{figure}[h]
	\begin{minted}[linenos]{OCaml}
type itype =
(* Poly is the supertype of all types represented as i32 in WebAssembly *)
| It_poly | It_bool | It_int | It_pointer | It_unit
| It_float
| It_none (* No type, used for functions with no return type *)
	\end{minted}
	\caption{An overview of the types used in the Intermediate Representation}
	\label{fig:ir}
\end{figure}
Translation into the IR involves translating each function's expression into a list of these intermediate instructions. The top-level AST is translated into it's own sequence of instructions and packaged into a new special `init' function which must take no arguments and return no results, hence the inclusion of the \camlinline{It_none} type. Throughout the translation, we keep track of variables introduced (both temporary and named) and their types. Named variables in the top-level AST become global variables, while temporary variables introduced here become local variables of the init function.
\\\\
This transformation is handled in \textinline{intermediate.ml}, with two main functions:
\begin{itemize}
	\item \camlinline{transform_expr} Recursively transforms an expression into a list of instructions, and a variable for the result of the expression.
	\item \camlinline{transform_pat} This takes a pattern and a variable, and generates a list of instructions that will both test the structure of the value is stored in the variable against the pattern, and destructure this value into new variables. For instance in \camlinline{let (3, a) = x} we will generate code to test the first element of x's tuple to ensure that it is 3, and assign the second element to the variable a.
\end{itemize}
The majority of the transformation is relatively straightforward, however a few cases required additional thought:

\subsection{Closures and (Mutually) Recursive Functions}
Recursive functions require their own closure to call themselves recursively, and the problem is made worse by mutually recursive functions that all require access to each others' closures. Mutually recursive definitions can occur inside expressions, and hence it is not always the case that these closures would be available as global values. This left two possible options:
\begin{enumerate}
	\item Create a new closure for the recursive function or its mutually recursive friends whenever a recursive call occurs, by using the closure variables passed in as arguments to the function.
	\item Include recursive closures inside their own `closure variables'.
\end{enumerate}
I chose the second option because the first option would have been both less efficient, and more difficult to implement. In order to include closures inside their own `closure variables', there are two instructions to create closures. The first one, \textinline{Iins_newclosure}, creates a `new' empty closure for the specified function, while \textinline{Iins_fillclosure} takes an existing empty closure, and fills it up using the variables provided, one of which can be the variable storing the currently empty closure or currently empty closures of mutually recursive friends.

\subsection{Match Statements}
Match statements are handled through nested blocks. The outer block represents the entire match statement, while inner blocks represent individual cases of the match statement. In addition, a temporary variable is introduced to store the result of the entire match statement. The code for an individual case is as follows
\begin{enumerate}
	\item Code for the case's pattern, modified to replace fail instructions (indicating the pattern was not matched), with instructions to jump out of the case block and thus proceed with the next case.
	\item Code for the case's expression
	\item An instruction to copy the expression's result into the match statement's result
	\item An instruction to jump out of the outer match statement block, and thus skip over the remaining cases
\end{enumerate}
In addition, a fail instruction is included after all the cases inside the match block to ensure that if no cases match, execution terminates with a failure.

\subsection{Polymorphism}
Polymorphism is implemented through the special intermediate type \camlinline{It_poly}, which indicates that a variable holds a polymorphic value. This type is used for the arguments and results of all closures, as when an arbitrary closure is called we cannot be sure whether it is a polymorphic function or not and thus must assume it is. It is also used for the contents of tuples, which can also be used polymorphically. 
\\\\
Types that are represented as 32-bit integers in WebAssembly, such as integers, units, booleans and pointers, are all subtypes of this polymorphic type, and hence values of these types can be directly stored in these variables and passed as arguments to functions. Floating point numbers however are not a subtype, as the assignment of a floating point value to an integer variable in WebAssembly is not allowed. Hence we have to `box' these floating-point values, meaning that we allocate a space in memory and store the floating-point value there, passing the pointer to this location instead.
\\\\
The result of the intermediate translation is a list of function objects, which contain the name, intermediate code, and list of variables used by that function. In addition, a list of global variables is produced.





\section{Optimisations on the IR}
%\note{
%	\begin{itemize}
%		\item Unreachable code elimination (needed due to tail-call optimiser's generation of broken unreachable code e.g. assign unit to float because the `break' statement / tail-call gives a unit type)
%		\item Copy propagation (if we have y=x, followed by z = y+y we can replace y with x. By far the most complex of the IR optimisations because we need to know both that the most recent definition of y is y=x, and that x has not changed since then)
%		\item Tuple load elimination (If we know a tuple is (x,y,z), then instead of loading it when matching with (a,b,c), we can just do a=x, b=y, z=c directly. Also works on constructs)
%		\item Dead code elimination (removes useless units, and also tuple creation when tuple usage eliminated by tuple load elimination)
%		\item Ref elimination (eliminates refs that are used as mutable variables only, by using mutable variables)
%	\end{itemize}
%}

Before code generation, a second round of optimisations is performed. These optimisations are predominantly data-flow analyses. Data-flow analysis concerns the movement of data through program code, asking questions like `Will the value assigned here be used?' (Live Variable Analysis) and `Where could the current value of this variable have been assigned?' (Reaching Definitions). Unlike the first round of optimisations, which improved execution time at the expense of code size, these optimisations both improve execution time and decrease code size.
\\\\
Before data-flow analysis can be performed, function code must first be deconstructed into a basic-block graph. A basic block is a sequence of instructions that will always begin execution at the first instruction, and finish at the last, with no jumps out in-between. A conditional jump at the end of one basic block might mean that there are several possible blocks that could be executed next, and hence a graph is required to keep track of these.
\\\\
My approach to construct this graph is to first build a jump-table which records which lines have jumps and where they might jump to. From this we can extract all locations with incoming or outgoing jumps, and use these to split the code into these blocks. We can then use the table to determine the outgoing edges of the basic blocks, and from those record in each basic block its possible predecessor or successor basic blocks.

\subsection{Data-Flow Analysis}
My compiler then implements three forms of data-flow analysis for use in optimisations:
\begin{itemize}
	\item Live Variable Analysis: For each instruction $i$, the variable $x$ is syntactically live at that instruction if there exists an instruction $j$ that is reachable from $i$ without passing any assignments to $x$ and uses the value of $x$. There is another definition of liveness, semantic liveness, whereby a variable is life if it's value will affect the input/output behaviour of the program at some point in the future. As this is generally uncomputable, syntactic liveness is used as a safe approximation.
	\item Reaching Definitions: For each instruction $i$, a set for each variable $x$ of instructions $j$ that assign to $x$ and for which there is a path from $j$ to $i$ without passing another assignment of $x$.
	\item `Available Assignments': For each instruction $i$, a set for each variable $x$ of variables $y$ that, if assigned in an instruction that uses the value of $x$, the value of $x$ has not changed since that assignment to $y$ (but $y$ itself may have been reassigned). This analysis is similar to the more conventional Available Expressions analysis, except that instead of storing expressions for which their operands have not changed since they were computed, it stores the variables assigned from these expressions.
\end{itemize}
An example of these analyses is shown in figure \ref{fig:dataflow}. Line 0 is used to refer to the beginning of the function. Note that on line 8, $y$ is still in the available assignment set for $x$. This is correct, as in all cases where $x$ is assigned to $y$ (as in line 3), $x$ is not modified after this assignment.
\begin{figure}[h]
	\begin{minipage}{0.3\linewidth}
	\begin{minted}[linenos, firstnumber=0]{python}
myfunc(x):
  y = 9
  if cond:
    y = x
    y = 10
  else:
    x = 2
    y = 4
  return y
	\end{minted}
	\end{minipage}
	\begin{minipage}{0.7\linewidth}
		\begin{center}
			\begin{tabular}{|c|c|c|c|}
				\hline
				Line & LVA (at end) & RDs (at start) & AAs (at start) \\
				\hline
				1 & $x$ & $(x, \{0\})$ & \\
				\hline
				3 & & $(x, \{0\}), (y, \{1\})$ & \\
				4 & $y$ & $(x, \{0\}), (y, \{3\})$ & $(x, \{y\})$ \\
				\hline
				6 & & $(x, \{0\}), (y, \{1\})$ & \\
				7 & $y$ & $(x, \{6\}), (y, \{1\})$ & \\
				\hline
				8 & & $(x, \{0, 6\}), (y, \{4, 7\})$ & $(x, \{y\})$ \\
				\hline
			\end{tabular}
		\end{center}
	\end{minipage}
	\caption{An example of the three data-flow analyses}
	\label{fig:dataflow}
\end{figure}
\\\\
For each of these analyses, it is possible to define data-flow equations where we compute these sets for a line by taking either the union or intersection of the sets from all lines that are direct predecessors or successors of the line. The equations for reaching definitions analysis are as follows:
\begin{align*}
\text{in-defs}(i) \quad=&\quad \bigcup_{p\ \in\ \text{pred}(i)} \text{out-defs}(p) \\
\text{out-defs}(i) \quad=&\quad (\text{in-defs}(i) \setminus \text{undef}(i)) \cup \text{def}(i)
\end{align*}
where $\text{undef}(i)$ and $\text{def}(i)$ depend on the instruction $i$. If $i$ does not assign a variable, they are both empty, otherwise if $i$ assigns to $x$, then $\text{undef}(i)$ is all previous assignments to $x$, while $\text{def}(i)$ is the new assignment to $x$, or $(x, \{i\})$. Similar equations can be given for the other analyses, using different def and undef functions: available assignments replaces the big-union with a big-intersection, and live-variables operates backwards using a union of the successor lines to compute the out-lva, and then computing in-lva from that.
\\\\
The algorithm for implementing these data-flow analysis follows from these equations: Put the set for each line to null initially, and iterate over all basic blocks applying these equations (and excluding nulls from intersections, an all null intersection gives the empty set) until there is an iteration in which nothing changes, and then return the result.

\subsection{Transformations From These Analyses}
These analyses allow me to implement a number of transformations:
\begin{itemize}
	\item Unreachable Code Elimination. Unreachable code is code that will never be executed, and hence it is safe to remove. A safe approximation of unreachable code is given by basic blocks that have no predecessors (and are not the first block of the function), and I eliminate the code in these blocks, taking care to retain structural instructions such as those that signify the end of an if-statement or `block'.
	\item Dead Code Elimination. Dead code is code that may be executed, but produces a result that will never be used. Live Variable Analysis can be used, as if a variable is not live at the end of a line where it is assigned, that assignment is never used, and thus if that line contains no side-effects it can be eliminated.
	\item Copy Propagation. Using the results of Reaching Definitions, it becomes possible to check when we see a use of a variable $y$, if the most recent definition of $y$ is $y = x$, and using Available Assignments, we can check that $x$'s value hasn't changed since that assignment. In this case, it is safe to replace a usage of $y$ with $x$. If all usages of $y$ are eliminated, the assignment $y = x$ can then be eliminated with Dead Code Elimination.
	\item Tuple-load Elimination. Similar to copy-propagation, if we encounter the loading of a variable from a tuple $t$, e.g. loading of the 2nd element of a 3 element tuple, we can use Reaching Definitions to see if there is a unique assignment of $t$. If there is and it is of the form $t = (x_1, ..., x_n)$, we can then check using Available Assignments to make sure the $x_i$ we want has not been modified since, and if so we replace the load with $x_i$. If this eliminates all usages of $t$, Dead Code Elimination can then eliminate the construction of the tuple, which can result in a significant speedup. This optimisation is particularly useful when dealing with match statements over multiple variables, as it eliminates the creation of the tuple in that case.
\end{itemize}

\subsection{Other Transformations}
In addition, I perform a transformation I call as `ref elimination'. If a reference is created inside a function, and then only used when it is updated or dereferenced, without being passed as an argument to another function, stored in memory or returned from the function, then the reference can be eliminated and instead a mutable variable is used, which eliminates the memory operations that go with using the reference, and results in a small but noticeable speedup.

\subsection{Ordering of Transformations}
Individually, each transformation is looped until it does not result in a change to the code. However, the order in which the transformations are performed can have a great effect on their effectiveness, for instance if Dead Code Analysis were run only before Tuple-load Elimination, we would not be able to eliminate the unused tuple assignments resulting from Tuple-Load Elimination.
\\\\
Therefore, I order the transformations by running each in order, but then running transformations that might benefit from an earlier transformation again after the earlier transformation. This results in some transformations being run many times, however I do not consider that a problem as the compiler is fast to execute for all samples.

\section{Code Generation}
\note{The stack code generator removes redundant variable saving/loading. Also representation of closures and tuples/constructs/refs}





\section{Summary}
\note{Summarise all that is done with a sentence per stage}





\section{Overview of the Files}







% CHAPTER
\clearpage
\chapter{Evaluation}
\note{This is where Assessors will be looking for signs of success and for evidence of thorough and systematic evaluation as discussed in Section 8.3. Sample output, tables of timings and photographs of workstation screens, oscilloscope traces or circuit boards may be included. A graph that does not indicate confidence intervals will generally leave a professional scientist with a negative impression.}

\note{As with code, voluminous examples of sample output are usually best left to appendices or omitted altogether.}

\note{There are some obvious questions which this chapter will address. How many of the original goals were achieved? Were they proved to have been achieved? Did the program, hardware, or theory really work?}

\note{Assessors are well aware that large programs will very likely include some residual bugs. It should always be possible to demonstrate that a program works in simple cases and it is instructive to demonstrate how close it is to working in a really ambitious case.}

\section{Success Criteria}
\note{Say that it has been satisfied, and list the success criteria}

\section{End To End Tester}
\note{Could do with a -all mode to test all combinations of disabled / enabled optimisations}

\section{Benchmarks / Benchmark Driven Optimisations}
\note{
	\begin{itemize}
		\item Gcd: main target of optimisations. Went from 122ms to 8ms thanks to all of them.
		\item others: they improve (e.g. especially with direct call generation), but not as much with the other optimisations
		\item Make some performance graphs of the final result vs other execution engines, and of different optimisations enabled/disabled (can do both time and memory for that).
	\end{itemize}
}

% TODO DONT FORGET ALL THE WORK IN MAKING THE CODE SHORTER

% TODO HOW IT WAS EVALUATED
% TODO TEST SYSTEM AND BENCHMARKS




\clearpage
\chapter{Conclusion}
\note{This chapter is likely to be very short and it may well refer back to the Introduction. It might properly explain how you would have planned the project if starting again with the benefit of hindsight. }

\note{
	\begin{itemize}
		\item SSA based IR?
		\item More optimisations e.g. reverse copy-propagation, mutually recursive tail call optimisation, better match statements, better code generator that can re-order non side effecting instructions, speedy append implementation
		\item More feature support: strings, records and mutable records, modules, named/optional function arguments, using the operators as function arguments (e.g. List.reduce ~f:(+) nums to sum a list)
	\end{itemize}
}

% TODO CONCLUDE THE DOCUMENT
% TODO IVE SUCCESSFULLY MET SUCCESS CRITERIA, SUMMARY OF EVERYTHING
% TODO WHAT LESSONS DID I LEARN
% TODO WHAT WORK COULD YOU DO IN THE FUTURE




\clearpage

%%%%%%%%%%%%%%%%%%%%%%%%%%%%%%%%%%%%%%%%%%%%%%%%%%%%%%%%%%%%%%%%%%%%%
% the bibliography

\addcontentsline{toc}{chapter}{Bibliography}
\printbibliography[title={Bibliography}]
\clearpage

%%%%%%%%%%%%%%%%%%%%%%%%%%%%%%%%%%%%%%%%%%%%%%%%%%%%%%%%%%%%%%%%%%%%%
% the appendices
\appendix
% Assessors like to see some sample code or example circuit diagrams, and appendices are the sensible places to include such items. Accordingly, software and hardware projects should incorporate appropriate appendices. Note that the 12,000 word limit does not include material in the appendices, but only in extremely unusual circumstances may appendices exceed 10-15 pages - if you feel that such unusual circumstances might apply to you you should ask your Director of Studies and Supervisor to apply to the Chairman of Examiners. It is quite in order to have no appendices. Appendices should appear between the bibliography and the project proposal. 

\chapter{The First Appendix}

Things in appendix A


\clearpage

\chapter{The Intermediate Representation}

\note{TODO FIX THE COMMENTS, SOME ARE FROM OLD IR}

\begin{minted}{OCaml}
type itype =
| It_poly | It_bool | It_int | It_pointer | It_unit
| It_float
| It_none

type iunop =
| Iun_neg (* Negate *)
| Iun_eqz (* Equals zero *)

type ibinop =
| Ibin_add | Ibin_sub | Ibin_mul | Ibin_div | Ibin_rem (* Arithmetic *)
| Ibin_and | Ibin_or (* Logic *)
| Ibin_eq | Ibin_ne (* Equality *)
| Ibin_lt | Ibin_le | Ibin_gt | Ibin_ge (* Comparison *)

type iscope = Local | Global (* Variable scope *)

type ivariable = iscope * string

type iinstruction =
(* Create a new var from a constant *)
(* type of variable, name of variable *)
| Iins_setvar of itype * ivariable * string
(* Copy a variable into another *)
(* type of variable, name of new variable, name of old variable *)
| Iins_copyvar of itype * ivariable * ivariable
(* Return var *)
(* type of variable, name of variable *)
| Iins_return of itype * ivariable
(* Unary operation using one stack value *)
(* type of operand, unary operation, result var, input var *)
| Iins_unop of itype * iunop * ivariable * ivariable
(* Binary operation using two stack values *)
(* type of operands, binary operation *)
| Iins_binop of itype * ibinop * ivariable * ivariable * ivariable
(* Make a new closure for specified function and tuple type, and put it in given variable *)
(* type of function, name of function, type of closure variables, variable to put closure in *)
| Iins_newclosure of iftype * string * ituptype * ivariable
(* Fill a closure in the named variable using the code to generate those values *)
(* type of closure variables, name of variable, list of variables to copy in *)
| Iins_fillclosure of ituptype * ivariable * ivariable list
(* Call closure in variable using argument generated from code *)
(* type of function, output variable, closure variable, variable for argument *)
| Iins_callclosure of iftype * ivariable * ivariable * ivariable
(* Directly call a function *)
(* output variable, name of function, type of args, arg vars *)
| Iins_calldirect of ivariable * string * ituptype * (ivariable list)
(* Start a block *)
(* name of block *)
| Iins_startblock of string
(* End a block *)
(* name of block *)
| Iins_endblock of string
(* Exit from the named block *)
(* name of block *)
| Iins_exitblock of string
(* Exit from the named block if variable is true *)
(* name of block *)
| Iins_exitblockif of string * ivariable
(* Start an if statement *)
(* name of block, condition var *)
| Iins_startif of string * ivariable
(* Else clause of an if statement *)
(* name of block *)
| Iins_else of string

(* End an if statement *)
(* name of block *)
| Iins_endif of string
(* Starts a loop, loops until an exitblock or exitblockif *)
(* Name of escape block (to break to), name of loop block (to continue to) *)
| Iins_startloop of string * string
(* Ends a loop *)
(* Name of break block, name of continue block *)
| Iins_endloop of string * string
(* Create a tuple from the given vars, push pointer to stack
 * and put it in that variable *)
(* type of tuple, name of variable, code to generate each part of tuple *)
| Iins_pushtuple of ituptype * ivariable * ivariable list
(* Pop tuple, push its value at index i to the stack *)
(* type of tuple, index in tuple, output var, tuple var *)
| Iins_loadtupleindex of ituptype * int * ivariable * ivariable
(* Create a construct from the given vars, push pointer to stack and put it in variable *)
(* type of construct arguments, name of variable, id of construct, arguments *)
| Iins_pushconstruct of ituptype * ivariable * int * ivariable list
(* Pop construct, push its value at index i to the stack *)
(* type of construct arguments, index in arguments, output variable, tuple variable *)
| Iins_loadconstructindex of ituptype * int * ivariable * ivariable
(* Pop construct, push its id to the stack *)
(* output variable, construct variable *)
| Iins_loadconstructid of ivariable * ivariable
(* Box a value (on stack) of a type that needs boxing, or for a ref *)
(* unboxed type, unboxped variable, variable to store boxed result *)
| Iins_newbox of itype * ivariable * ivariable
(* Update a boxed value, useful for refs *)
(* unboxed type, unboxped variable, boxed variable *)
| Iins_updatebox of itype * ivariable * ivariable
(* Unbox a value / dereference a ref *)
(* unboxped type, wrapped variable, unboxped target variable *)
| Iins_unbox of itype * ivariable * ivariable
(* Fail *)
(* No parameters *)
| Iins_fail
\end{minted}


\clearpage

\chapter{Project Proposal}
\clearpage

\thispagestyle{empty}
	
	%\rightline{\large{Paul Durbaba}}
	%\medskip
	%\rightline{\large{Robinson}}
	%\medskip
	%\rightline{\large{pd452}}
    \rightline{\large{Candidate 2376D}}
	
	\vfil
	
	\centerline{\large Computer Science Tripos: Part II Project Proposal}
	\vspace{0.4in}
	\centerline{\Large\bf Compiling OCaml to WebAssembly}
	\vspace{0.3in}
	\centerline{\large{Friday 18\textsuperscript{th} October, 2019}}
	
	\vfil
	
	{\bf Project Originator:} Based on a proposal by Timothy Jones
	
	\vspace{0.1in}
	
	{\bf Resources Required:} See attached Project Resource Form
	
	\vspace{0.5in}
	
	{\bf Project Supervisor:} Tobias Kohn
	
	\vspace{0.2in}
	
	{\bf Signature:}
	
	\vspace{0.5in}
	
	{\bf Director of Studies:}  Alan Mycroft
	
	\vspace{0.2in}
	
	{\bf Signature:}
	
	\vspace{0.5in}
	
	{\bf Overseers:} Pietro Lio' and Robert Mullins
	
	\vspace{0.2in}
	
	{\bf Signatures:} 
	
	\vfil
	\eject
	
	
	\section*{Introduction and Description of the Work}
	The aim of the project is to implement a compiler from a subset of OCaml to WebAssembly.
	\\\\
	WebAssembly\footnote{https://webassembly.org/} is a binary instruction format for the web with the main goal of improving performance of more computationally intensive functions in web applications. It does not replace JavaScript as there (currently) is no way to perform tasks such as DOM manipulation directly from WebAssembly - it is expected that a JavaScript application might call some functions implemented in WebAssembly to perform computation, and then display the results itself.
	\\\\
	My work will involve writing a compiler in OCaml and a runtime system in a language such as C. 
	\\\\
	The compiler will take an AST (abstract syntax tree) produced by the lexer and parser of the OCaml compiler\footnote{https://github.com/ocaml/ocaml} (ocaml-compiler-libs), performing type checking (I considered using the OCaml type checker, but getting that to work might be just as hard as writing my own type checker), and then transforming the typed AST through a series of intermediate representations (IRs). Each IR will eliminate some higher-level feature, for instance replacing first-class functions with closures. At the end, I will generate instructions to target WebAssembly. The WebAssembly Binary Toolkit\footnote{https://github.com/WebAssembly/wabt} will then be used to output a usable WebAssembly module.
	\\\\
	The runtime system will be needed to implement closures and memory allocation / access in a way that can be used by the code generated by my compiler.
	
	\section*{Starting Point}
	\subsection*{OCaml and Compilers}
	I have done very little programming in OCaml previously, with the exception of coming up with some code samples when considering which features this compiler should support. However I have some experience in programming in Standard ML, a similar language, from Part IA Foundations of Computer Science.
	\\\\
	I also have little previous experience in writing compilers. I have written basic lexers and parsers, and implemented a system for interpreting mathematical expressions, however I have no experience with a larger compiler that compiles to instructions instead of interpreting.
	
	\subsection*{WebAssembly and the Web}
	I have no experience with WebAssembly with the exception of reading through some of the documentation in the weeks leading up to this proposal.
	\\\\
	I have a fair amount of experience with JavaScript and websites: I do not anticipate any issue with coming up with a suitable demonstration of the WASM produced by the compiler.
	
	
	\section*{Substance and Structure of the Project}
	
	%KEY CONCEPTS, MAJOR WORK ITEMS, THEIR RELATIONS AND RELATIVE IMPORTANCE, DATA STRUCTURES AND ALGORITHMS
	
	%Key Concepts: Type Checking, WebAssembly, Runtime System, Closure
	%Major Work Items: Components
	%Relative importance: All important!
	
	%Data Structures: Trees, lots of trees!

	The main components of the compiler are the type checker, the intermediate representations, the WebAssembly code generator, and the runtime system. These will need to be implemented in the order listed, as each depends on the last, with the exception of the runtime system and code generator which can be developed in parallel, as a change in one might require a change in the other. Likewise, they are all equally important in a complete compiler, as each part is required during the translation from AST through to a WebAssembly module.
	\\\\
	My subset of OCaml will include the following:
	\begin{itemize}
		\item Values: \camlinline{int} (64-bit signed integer), \camlinline{bool} and \camlinline{float} (64-bit floating point), able to be defined using expressions \camlinline{let} and \camlinline{let ... in}.
		\item Functions: \camlinline{let} and \camlinline{let rec} with multiple curried arguments, with an expression for the function body, as well as inline functions using \camlinline{fun}
		\item Types: Non-polymorphic types defined using \camlinline{type}, and tuples. These can be used to implement list types, and hence lists themselves will not be part of the subset (although supporting list syntax will be an extension)
		\item \camlinline{if ... then ... else} and basic pattern matching. I was planning on excluding pattern matching from the initial subset, however it is needed to match types and de-structure them, and alternate approaches would lead to clumsy syntax. Pattern matching will be limited to use on types.
	\end{itemize}
	Polymorphism is excluded from the initial subset as I do know how challenging to implement it will be, however it will likely be the first stretch goal that I try and implement.

	\subsection*{Testing}
	I intend to write unit tests, for instance using the OUnit\footnote{https://github.com/gildor478/ounit} unit testing framework. These will function to demonstrate that sub-components are working before the whole compiler can be tested, and more importantly to prevent the re-introduction of bugs that have already been fixed by writing a unit test when a bug is found to ensure it doesn't occur again.
	\\\\
	The overall compiler will be tested with a selection of code samples designed to test important features and edge cases. The compiled code will be executed in the browser and the result checked against the expected result. This could also be automated, e.g. by using a WebAssembly interpreter such as the WebAssembly reference interpreter\footnote{https://github.com/WebAssembly/spec/tree/master/interpreter}, depending on the complexity of this.
	
	\subsection*{Evaluation}
	I will evaluate my compiler in comparison to other methods of executing OCaml, particularly on the web, such as the OCaml compiler and Js\_of\_ocaml\footnote{https://ocsigen.org/js\_of\_ocaml/3.1.0/manual/overview}. I will do this by measuring performance in terms of how long particular samples of OCaml code take to execute on these different platforms, by running them multiple times on each platform (or the sample includes a loop of the same effect), and recording the time between execution starting and execution completing.
	\\\\
	It would also be interesting to compare memory usage between the different approaches, however that will likely be unfeasible due to the difficulty of getting accurate memory usage statistics, e.g. how much of the browser's memory is used by the JavaScript OCaml interpreter.
	\\\\
	When optimisations are implemented, I will analyse the performance with and without the optimisation, for instance by using timing facilities available in JavaScript. An optimisation will be considered successful if it leads to a measurable increase in execution speed (or a measurable decrease in memory usage) for a carefully chosen set of OCaml programs.
	
	
	\section*{Success Criterion}
	The project will be a success if the following criteria are met:
	\begin{itemize}
		\item The compiler can take a selection of reasonable (not likely to result in more than 10MB of memory allocation) samples written using all features of the selected subset of OCaml, and compile them to WebAssembly modules
		\item These WebAssembly modules can be loaded and executed with a set of provided input values for each sample in the latest version of Chrome and Firefox to produce the same results as if the OCaml code they were compiled from was compiled, executed, and given the same input value by the OCaml Compiler
	\end{itemize}
	
	
	\section*{Possible Extensions}
	% Pattern Matching, References and Loops, Garbage Collection, Exceptions,
	\begin{itemize}
		\item Expanding my subset of OCaml supported: polymorphism, more advanced pattern matching, list syntax, strings, references, for and while loops, modules...
		
		\item Improving code access from JavaScript, for instance allowing JavaScript objects (probably just numbers and strings) to be passed to OCaml functions, likely with a special type to represent JavaScript objects
		
		\item Optimisations: There are many optimisations that could be implemented to improve the code generated by the compiler, for instance peephole optimisation, dead-code elimination, constant propagation, function call inlining, and many more that I could learn about.
		
		\item End to End testing: Using a WebAssembly interpreter, whereby I can specify a sample of programs to compile, and the expected output from running them, and validate that they all compile and produce the expected results in the interpreter.
		
		\item Supporting Exceptions. This would be challenging due to WebAssembly's current lack of support for them, however there are ways of working around this, with their own performance costs. One way might be to make functions that throw exceptions return a type that signifies either a result or an exception was thrown, then we can check if an exception was thrown after each function call, and propagate exceptions until a relevant catch block is found.
		
		\item Garbage Collection. WebAssembly likewise does not currently support garbage collection, and additionally does not allow walking the stack for security reasons. There are limited ways to work around this: by storing all values except for primitive types and references on the heap, reference counting could be used to track which heap items have references on the stack, and tracing could be used between heap items. A system like this would likely have to be implemented in JavaScript or TypeScript due to the complexity of getting it right in WebAssembly.
		
	\end{itemize}
	
	\section*{Timetable and Milestones}
	
	\subsection*{25th Oct - 8th Nov}
	Getting ready:
	\begin{itemize}
		\item Creating a git repository and GitHub remote for the project, and skeleton for the overall project
		\item Setting up the required libraries (the ocaml-compiler-libs)
		\item Setting up and trying out unit testing with OUnit
		\item Parsing OCaml files using the ocaml-compiler-libs parser, and experimenting with the resulting ASTs to learn more about them 
	\end{itemize}
	In addition, I will practice coding in OCaml to learn more about the language and come up with useful code samples that will be useful for testing later on. The book \textit{Types and Programming Languages} by Benjamin C. Pierce contains some OCaml samples to help me learn and also prepare me for writing the type checker.
	
	\subsection*{9th Nov - 22nd Nov}
	I will implement a type checker for the initial subset of OCaml. If I have time, polymorphism (an extension) could also be implemented at this stage, certainly my implementation of type checking will have to consider how polymorphism will be implemented in the future.
	\\\\
	At the end of this I should have a function that can take an OCaml AST and output a typed AST, with a type attached to each node.
	
	\subsection*{23nd Nov - 6 Dec}
	Implement translation of the typed AST through to a lower-level representation suitable for transformation into WebAssembly instructions.
	
	\subsection*{7 Dec - 20 Dec}
	Implement code generation and a suitable runtime system for WebAssembly to perform memory allocation and closure calls, probably using C. After this point, it should be possible to use the compiler to compile some code samples and execute them in a browser, and hence I will have met my success criterion.
	
	\subsection*{20 Dec - 3 Jan}
	I will probably be taking about a week's break from the project, and using the remaining time in these two weeks to clean up / improve the project if the success criterion has been met (for instance by refactoring, implementing additional test cases or adding better error messages), and continue working up to the success criterion otherwise.
	
	\subsection*{4 Jan - 17 Jan}
	I will implement polymorphism in the type checker and some form of polymorphism elimination in the intermediate representation. I also hope to implement references and for/while loops, which should be fairly simple to implement but will allow more interesting programs to be compiled.
	
	\subsection*{17 Jan - 31 Jan}
	I will implement some optimisations such as the ones I listed above, and evaluate them as I described in the evaluation section.
	\\\\
	Write the progress report.
	
	\subsection*{1 Feb - 14 Feb}
	Presentations and preparation needed for them.
	\\\\
	I will investigate and implement some way of passing JavaScript values (numbers and strings) to functions, and seek to improve my pattern matching implementation by compiling more complicated match statements.
	
	\subsection*{15 Feb - 28 Feb}
	I will set up an end to end testing system as described in the stretch goals section.
	
	\subsection*{1 Mar - 13 Mar}
	Start writing dissertation. Investigate supporting exceptions and garbage collection - perhaps the WebAssembly extensions to support them will be finished by this time. I will decide to work on either exceptions, garbage collection or additional optimisations, depending on the complexity of supporting exceptions or garbage collection.
	
	\subsection*{14 Mar - 27 Mar}
	Continue writing dissertation. Attempt to wrap up remaining unfinished features e.g. those I started in the last two weeks.
	
	\subsection*{28 Mar - 10 Apr}
	Continue with the dissertation, submitting first drafts of chapters to my supervisor. Focus on improving the code e.g. by refactoring, or implementing additional tests.
	
	\subsection*{10 Apr - 24 Apr}
	Fully focus on the dissertation, leaving the code as it is. Submit draft version of dissertation to supervisor and DoS.
	
	\subsection*{24th Apr - 8 May}
	At this point the dissertation should be complete, with only minor changes to be made over these weeks.
	\\\\
	Towards the end of these two weeks, I will submit the final version of my dissertation.
	
	
	\section*{Resource Declaration}
	I plan to do the work via remote desktop to my server located in France, so I can easily switch between using my desktop and laptop. This has 32GB RAM and two 2 TB HDDs in RAID 1 (mirrored). I will use git for version control of code and important documents, which will be regularly pushed to a private GitHub repository. In addition, both my laptop and desktop automatically download backups of data on my server at regular intervals when they are online.
	\\\\
	I accept full responsibility for this machine and I have made contingency plans to protect myself against hardware and/or software failure.
	\\\\
	I will be making use of some components of the OCaml compiler (the lexer, parser, and possibly type-checker), as well as the WebAssembly Binary Toolkit (WABT).

\end{document}
